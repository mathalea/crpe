\usepackage[left=1.5cm,right=1.5cm,top=2cm,bottom=2cm]{geometry}
\usepackage[utf8]{inputenc}		        % Accents, encodage utf8
\usepackage[T1]{fontenc}		        % Encodage des caractères
\usepackage{lmodern}			        % Choix de la fonte (Latin Modern de D. Knuth)
\usepackage{mathtools}
\usepackage{multicol} 					% Multi-colonnes
\usepackage{calc} 						% Calculs 
\usepackage{enumitem}					% Pour modifier les numérotations
\usepackage{graphicx}					% Pour insérer des images
\usepackage{tabularx}					% Pour faire des tableaux
\usepackage{pgf,tikz}					% Pour les images et figures géométriques
\usepackage{cancel}						% Pour pouvoir barrer les nombres
\usepackage{url} 			        	% Pour afficher correctement les url
 \urlstyle{sf}                          % qui s'afficheront en police sans serif
\usepackage{eurosym}					% Pour utiliser la commande \euro
\usepackage{xlop}						% Pour les calculs posés
\usepackage{colortbl} 					% Pour des tableaux en couleur
\usepackage{setspace}					% Pour \begin{spacing}{2.0} \end{spacing}
\usepackage{multirow}					% Pour des cellules multilignes dans un tableau
\usepackage{diagbox}					% Pour les cellules de tableau coupées en deux
\usepackage{textcomp, gensymb}			% Pour \degree - textcomp doit être importé avant gensymb pour éviter un warning
\usepackage{scratch3}					% Pour les blocs
\usepackage{tkz-euclide}				% Pour la géométrie
\usepackage[french]{babel}	        	% Les règles de typo. françaises


				
\setlength{\parindent}{0mm}				% Pas de retrait en début de paragraphe

\setlist[enumerate]{itemsep=1em}
\setlist[enumerate, 1]{font=\bfseries}
\setlist[enumerate, 2]{label=\alph*., font=\bfseries}




 