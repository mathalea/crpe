Célia s'entraîne à courir tous les jours de la semaine sur le même parcours.
\begin{enumerate}
    \item Elle aimerait comparer ses résultats d'entraînement sur une semaine à ceux de sa s\oe ur qui s’entraîne également sur le même parcours.
    
    \medskip
	\noindent\begin{tabularx}{0.4\linewidth}{|X|}
        \hline 		
        Résultats obtenus par Célia cette semaine :\\
        \\
        Lundi : 33 min et 12 secondes\\
        Mardi : 32 min et 4 secondes\\
        Mercredi : 40 min et 25 secondes\\
        Jeudi : 27 min et 11 secondes\\
        Vendredi : 30 min\\
        Samedi : 26 min et 38 secondes\\
        Dimanche : 29 min et 1 secondes\\
        \hline 
    \end{tabularx}
    \hspace{1cm}
    \noindent\begin{tabularx}{0.45\linewidth}{|X|}
        \hline 		
        Résultats obtenus par sa s\oe ur cette semaine :\\
        \\
        Moyenne : 31 min et 13 secondes\\
        Médiane : 30 min\\
        Étendue : 3 min\\
        \\
        \\
        \\
        \\
        \hline 
    \end{tabularx}
    \medskip
    \begin{enumerate}
        \item Comparer les durées moyennes de course.
        \item Comparer les durées médianes de course.
        \item Avec les informations ci-dessus, Célia affirme « Je suis la seule de nous deux à avoir
        réussi à effectuer ce parcours en moins de 28 minutes cette semaine ». Cette affirmation est-elle vraie ?
        \item Avec les informations ci-dessus, sa s\oe ur lui répond « Moi, j’ai été la plus régulière de
        nous deux sur la semaine~». Expliquer ce commentaire.
    \end{enumerate}
    \item Le parcours d’entraînement de Célia est représenté ci-contre.
    
    \begin{minipage}{5cm}
        Le diamètre $[AB]$ du demi-cercle reliant le point $A$ au point $B$ a pour longueur 2 300 m.
    \end{minipage}
     \hspace{1cm}
    \begin{minipage}{10cm}
        \begin{tikzpicture}[x=1cm,y=1cm]            
            \draw (0,0) arc (180:0:4cm);
            \draw (0,0) -- (8,0);            
            \draw (-0.2,-0.2) node {$A$};            
            \draw (8.2,-0.2) node {$B$};
        \end{tikzpicture}
    \end{minipage}
    \medskip
    \begin{enumerate}
        \item Représenter le parcours à l’échelle $\dfrac{1}{20~000}$. Justifier les mesures retenues pour réaliser
        la construction à l’échelle.
        \item Montrer que la distance du parcours, arrondie à l’unité, est d’environ 5 913 m.
        \item Aujourd’hui, Célia a bouclé le parcours sur une durée de 33 minutes et 36 secondes.
                
        Quelle a été sa vitesse moyenne en km/h, arrondie au dixième près ?
        \item Célia a l'habitude d'effectuer le parcours dans le sens des aiguilles d'une montre en
        partant du point $A$. Sur la représentation de la question \textbf{2.a.}, placer les points $L$, $M$ et
        $N$ correspondants respectivement au quart, à la moitié et aux trois quarts du parcours.
    \end{enumerate}
\end{enumerate}