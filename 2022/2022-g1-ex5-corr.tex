\begin{enumerate}
	\item 
	\begin{enumerate}
		\item Volume du cône : $V_c = \dfrac{1}{3} \times \pi \times r^2 \times h = \dfrac{1}{3} \times \pi \times 30^2 \times 90 = 27~000~\pi~\text{cm}^3$.
		
		\medskip
		Volume de la demi-boule : $V_b = \dfrac{1}{2} \times \dfrac{4}{3} \times \pi \times r^3 = \dfrac{2}{3} \times \pi \times 30^3 = 18~000~\pi~\text{cm}^3$.
		
		\medskip
		Volume total : $V_t = V_c + V_b = 27~000~\pi~\text{cm}^3 + 18~000~\pi~\text{cm}^3 = 45~000~\pi~\text{cm}^3$
		
		\item $45~000~\pi~\text{cm}^3 = 45~\pi~\text{dm}^3=45~\pi~\text{L}\approx 141~\text{L}$ (à l'entier près).
	\end{enumerate}
	
	\item On sait que le triangle $SON$ est un triangle rectangle en $O$. Donc d'après le théorème de Pythagore :\\
	 $SN^2 = SO^2 + ON^2$\\
	 $SN^2 = 30^2 + 90^2 = 9~000$ donc $SN = \sqrt{9~000}~\text{cm}$.
	 
	 \item L'aire totale est la somme de l'aire de la demi-sphère et celle de la surface latérale du cône de révolution.
	 
	 \medskip
	 $A_\text{totale} = \pi \times 30 \times \sqrt{9~000} + \dfrac{1}{2} \times 4\times\pi \times 30^2 \approx = 14~596~\text{cm}^2$ (à l'unité près).\\
	 Soit une aire d'environ 1,5~m\up{2} (au dixième près).
	 
	 \item
	 \begin{enumerate}
	 	\item Les longueurs sont multipliées par $1,25$.
	 	
	 	\item Dans un agrandissement de coefficient $1,25$, les aires sont multipliées par $1,25^2$.\\
	 	L'aire du ballon sonde à 4500~m d'altitude est donc : $1,5~\text{m}^2\times 1,25^2 \approx 2,3~\text{m}^2$.
	 	
	 	\item Dans un agrandissement de coefficient $1,25$, les volumes sont multipliées par $1,25^3$.\\
	 	$45~\pi~\text{L} \times 1,25^3 \approx 276~\text{L}$ (au litre près).
	 \end{enumerate}
	 
	 \item On cherche les nombres $a$ et $b$tels que : $t(x) = ax + b$.\\
	 On sait que $t(0) = b = 15$ et que $t(4~500) = 4~500a + 15 = -12$ donc $4~500a=-27$ d'où $a=\dfrac{-27}{4~500}=-0,006$.
	 
	 \medskip
	 Finalement $t(x) = -0,006x + 15$.
	 
	 \item On cherche $x$ tel que : 
	 
	 \begin{spacing}{2}
     $-0,006x + 15 \leq 0$ (on soustrait 15 aux deux membres)\\
	 $-0,006x \leq -15$ (on divise les deux membres par $-0,006$ qui est négatif, on change le sens de l'inégalité)\\
	 $x \geq \dfrac{-15}{-0,006}$\\
	 $x \geq 2~500$	
	 \end{spacing}
	 
	 
	 
	 \medskip
	 À partir de 2~500~m la température devient négative.
	 
	 \item Le ballon se trouve à une altitude de 5~000~m lorsque la température a baissé de 30\degree C et atteint $-15$\degree C.
\end{enumerate}