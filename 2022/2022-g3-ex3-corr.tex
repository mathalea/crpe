\subsection*{Partie A : Installation du potager}
\begin{enumerate}
    \item D'une part $BC^2 = 26^2 = 676$
    
    D'autre part $AB^2+AC^2 = 10^2+24^2 = 100+576 = 676$

    Donc $BC^2=AB^2+AC^2$ d'après la réciproque du théorème de Pythagore le triangle $ABC$ est rectangle en $A$.

    \item 
    \begin{enumerate}
        \item On sait que $ADEF$ est un rectangle.
        
        Or, si un quadrilatère est un rectangle alors ses côtés opposés sont parallèles.

        Donc $(DE)$ est parallèle à $(AF)$ donc $(DE)$ est parallèle à $(AC)$.

        On sait que les points $B$ ; $D$ ; $A$ sont alignés ainsi que les points $B$ ; $E$ ; $C$.
        
        De plus $(DE)$ est parallèle à $(AC)$, d’après le théorème de Thalès, on a donc :

        \medskip
        $\dfrac{BD}{BA}=\dfrac{BE}{BC}=\dfrac{DE}{AC}$ soit $\dfrac{24-4,8}{24}=\dfrac{BE}{26}=\dfrac{DE}{10}$ donc $DE = \dfrac{19,2\times 10}{24}=8~\text{m}$.

        \item $\text{Aire}_{ADEF} = AD\times DE = 4,8\times 8 = 38,4~\text{m\up{2}}$
        
    \end{enumerate}
    \item 
    \begin{enumerate}
        \item $AD=x$ donc $BD = 24-x$.
        
        Par le même raisonnement, $\dfrac{BD}{BA}=\dfrac{BE}{BC}=\dfrac{DE}{AC}$ soit $\dfrac{24-x}{24}=\dfrac{BE}{26}=\dfrac{DE}{10}$

        Donc $DE = \dfrac{(24-x)\times 10}{24} = \dfrac{24\times 10}{24}-\dfrac{x\times 10}{24} = 10 - \dfrac{10}{24}x = 10 -\dfrac{5}{12}x$.

        \item $\text{Aire}_{ADEF} = AD\times DE = x\times \left(10-\dfrac{5}{12}x\right)$.
    \end{enumerate}
    \item 
    \begin{enumerate}
        \item Graphiquement, si la longueur $AD$ vaut $5$ m l'aire vaut environ $40$ m\up{2}.
        \item Si l'aire vaut $45$ m\up{2} la longueur $AD$ vaut environ $6$ m ou $18$ m.
        \item L'aire du potager est-elle supérieure ou égale à $50$ m\up{2} si $7\geq AD \geq 17$.
        \item L'aire maximale est, par lecture graphique, d'environ $60$ m\up{2} pour une longuueur $AD$ environ égale à $12$ m.
        
        $DE=10-\dfrac{5}{12}x = 10-\dfrac{5}{12}\times 12 = 10-5 = 5$ m.

        \textit{On peut également calculer $DE$ ainsi : $DE=\dfrac{60}{12}=5$ m.}
    \end{enumerate}
\end{enumerate}

\subsection*{Partie B : Choix du terreau}
\begin{enumerate}
    \item Je calcule le volume de terreau :
    
    $V=\dfrac{1}{3}\times 12~\text{m}\times 5~\text{m}\times 30~\text{m} = 6~\text{m\up{3}}$

    \item Tarif avec le magasin 1 : $T_1 = 20+0,1\times 6~000 = 620$ \euro
    
    Tarif avec le magasin 2 : Il faut $6~000$ L de terreau soit 300 sacs de 20 L    
    
    $T_2 = (300\times 2,35+10)\times 0,8 = 572$ \euro

    Tarif avec le magasin 3 : Il faut $6~000$ L de terreau soit 120 sacs de 50 L
    
    $T_3 = 120\times 5,37 = 644,4$ \euro

    Le tarif le plus interessant est le tarif2.

\end{enumerate}

\subsection*{Partie C : Plantation des fleurs}
\begin{enumerate}
    \item $\dfrac{90}{100}\times 20 = 18$ donc un élève peut espérer voir pousser 18 fleurs.
    \item $26\times 20 = 520$, il faut donc 520 graines pour l’ensemble de la classe.
    
    Un paquet contient $50$ graines, $520 = 10\times 50 + 20$
    
    Il faut donc 11 paquets de graines. 

    $11\times 4,53~\text{\euro}=49,83~\text{\euro}$

    Le budget à prévoir est de $49,83~\text{\euro}$

    \item 
    \begin{enumerate}
        \item $\dfrac{1}{6} +\dfrac{1}{8} = \dfrac{4}{24}+\dfrac{3}{24}=\dfrac{7}{24}$
        
        $\dfrac{7}{24}>\dfrac{6}{24}$ soit $\dfrac{7}{24}>\dfrac{1}{4}$ 
        
        Donc les bulbes représentent plus de 25 \% du potager.

        \item La proportion de bulbes de jonquilles dans le panier est de $\dfrac{5}{6}$.
        
        Donc la proportion de bulbes de tulipes est de $\dfrac{1}{6}$.
        
        Il y a donc $5$ fois moins de tulipes.
        
        S'il y a $30$ bulbes de jonquilles, alors le nombre de bulbes de tulipes est de $6$.

        \medskip
        \textit{On peut également appeler $x$ le nombre total de bulbes.}
        
        \textit{On a alors $\dfrac{5}{6}x = 30$ donc $x=30\times \dfrac{6}{5}$ soit $x=36$ et $36-30=6$}        
    \end{enumerate}
\end{enumerate}
