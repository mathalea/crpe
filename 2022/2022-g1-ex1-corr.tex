\subsection*{Partie 1}

\begin{enumerate}
	\item 
	\begin{enumerate}
		\item Un élève effectue 4 tours de 250 m soit : $4\times250~\text{m}=1~000~\text{m}$.
		
		Il effectue 1~000 m en 10 min. Sa vitesse moyenne en mètre par minute est donc : $v=\dfrac{1~000~\text{m}}{10~\text{min}}=100~\text{m/min}$.
		
		\item Un deuxième élève effectuer les 4 tours à une vitesse moyenne de 150 m/min. En 1 heure, il parcours donc\\ $150~\text{m/min} \times 60~\text{min} = 9~000~\text{m}$.\\
		$9~000~\text{m} = 9~\text{km}$ donc sa vitesse moyenne est de $9~\text{km/h}$.
		
		\end{enumerate}
		
		\item \begin{itemize}
			\item \textbf{Pour l'élève de CM1} : la distance parcourue est de $4 \times 400~\text{m} = 1~600~\text{m}$ ; la durée du parcours est de 9,5 min. La vitesse moyenne est donc de : $v = \dfrac{1~600~\text{m}}{9,5~\text{min}} \approx ≈168~\text{m/min}$⁡ (à l'unité près).
			
			\item \textbf{Pour l'élève de CM2} : la distance parcourue est de $4 \times 500~\text{m} = 2~000~\text{m}$ ; la durée du parcours est de\\ $11~\text{min}~8~\text{s} = 11~\text{min} + \dfrac{8}{60}~\text{min} = \dfrac{668}{60}~\text{min}$. La vitesse moyenne est donc de : $v = \dfrac{2~000~\text{m}}{\dfrac{668}{60}~\text{min}} \approx ≈180~\text{m/min}$⁡ (à l'unité près).
		\end{itemize}
	\end{enumerate}
	
\subsection*{Partie 2}
	
\begin{enumerate}
	\item 
	\begin{enumerate}
		\item La longueur du tour de pénalité est un cercle de rayon $R$ de 20 m de longueur on en déduit : \\
			$2\pi R =20~\text{m}$ soit $R = \dfrac{20~\text{m}}{2\pi} \approx 3,18~\text{m}$ (au cm près).
		
		\item \begin{itemize}
			\item \textbf{Pour le premier tour} : L'élève court à la vitesse de 150 m/min et doit effectuer 250~m soit un temps de course de : $t=\dfrac{250~\text{m}}{150~\text{m/min}}=\dfrac{5}{3}~\text{min}=1~\text{min}~40~\text{s}$.
			
			\medskip
			Donc la durée totale pour le premier tout est $t_1 = 1~\text{min}~40~\text{s} + 30~\text{s}=2~\text{min}~10~\text{s}$
			
			\medskip
			\item \textbf{Pour le deuxième tour} : La durée de course est toujours égale à 1~min~40~s sur le grand tour.\\
			Pour le tour de pénalité, il court à une vitesse moyenne de 150~m/min et doit effectuer 20~m soit un temps additionnel égal à : $t=\dfrac{20~\text{m}}{150~\text{m/min}}=\dfrac{2}{15}~\text{min}=\dfrac{2}{15}\times60~\text{s}=8~\text{s}$.
			
			Donc la durée totale pour le deuxième tour est $t_2=1~\text{min}~40~\text{s}+8~\text{s}+30~\text{s}=2~\text{min}~18~\text{s}$.
			
			\medskip
			\item \textbf{Pour le troisième tour} : $t_3=1~\text{min}~40~\text{s} + 2 \times 8~\text{s}+30~\text{s}=2~\text{min}~26~\text{s}$.
			
			\item \textbf{Pour le quatrième tour} : $t_4 = 1~\text{min}~40~\text{s}$.
		\end{itemize}
		
		\medskip
		La durée totale du parcours est de : $t=2~\text{min}~10~\text{s} + 2~\text{min}~18~\text{s} + 2~\text{min}~26~\text{s} + 1~\text{min}~40~\text{s} = 7~\text{min}~94~\text{s}=8~\text{min}~34~\text{s}$.
		\end{enumerate}
		
		\item
		\begin{enumerate}
			\item C3+E3+G3 nous donne le nombre de tirs manqués sur les tours et donc par conséquent le nombre de pénalités (nombre de petits tours à parcourir). Ce nombre est multiplié par 20 qui est la longueur d'un petit tour. Donc (C3+E3+G3)*20 est la longueur en mètres à parcourir en pénalité.
			
			\item Dans la colonne J, on calcule la vitesse moyenne en m/min. La formule à entrer est : $v=\dfrac{d~\text{(en m)}}{t~\text{(en min)}}$ donc J3=H3/(I3/60).
			
			\item K3=(I3+B3+D3+F3)/60
			
			\item Entre l'essai 2 et l'essai 3, l'élève a gagné du temps sur les tirs. Il a manqué davantage de tirs et a eu plus de pénalités.
			
			\item La stratégie à adopter est donc de s'appliquer sur les tirs pour en rater le moins possible et avoir le moins de pénalités possible.
			
		\end{enumerate}
\end{enumerate}