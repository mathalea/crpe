On peut résumer la situation par le tableau suivant : 
\begin{center}
\noindent\begin{tabularx}{\linewidth}{|c|*{6}{>{\centering\arraybackslash}X|}}
\hline 
\diagbox{Dé 2}{Dé 1} & 1 & 1 & 2 & 3 & 4 & 5\\ 
\hline 
1 & 2 & 2 & 3 & 4 & 5 & 6\\ 
\hline 
2 & 3 & 3 & 4 & 5 & 6 & 7\\ 
\hline 
3 & 4 & 4 & 5 & 6 & 7 & 8\\ 
\hline 
4 & 5 & 5 & 6 & 7 & 8 & 9\\ 
\hline 
5 & 6 & 6 & 7 & 8 & 9 & 10\\ 
\hline 
5 & 6 & 6 & 7 & 8 & 9 & 10\\ 
\hline 
\end{tabularx} 
\end{center}

\begin{enumerate}
	\item 
	\begin{enumerate}
		\item Les différentes sommes qu’un élève peut obtenir sont \textbf{2 ;3 ;4 ;5 ;6 ;7 ;8 ;9 ;10}.
		\item 
\begin{center}
\noindent\begin{tabularx}{\linewidth}{|c|*{6}{>{\centering\arraybackslash}X|}}
\hline 
\diagbox{Dé 2}{Dé 1} & 1 & 1 & 2 & 3 & 4 & 5 \\ 
\hline 
1 & 2 & 2 & 3 & 4 & 5 & 6 \\ 
\hline 
2 & 3 & 3 & 4 & 5 & 6 & 7 \\ 
\hline 
3 & 4 & 4 & 5 & 6 & 7 & \cellcolor{gray!30}{8} \\ 
\hline 
4 & 5 & 5 & 6 & 7 & \cellcolor{gray!30}{8} & 9 \\ 
\hline 
5 & 6 & 6 & 7 & \cellcolor{gray!30}{8} & 9 & 10 \\ 
\hline 
5 & 6 & 6 & 7 & \cellcolor{gray!30}{8} & 9 & 10 \\ 
\hline 
\end{tabularx} 
\end{center}	
La probabilité de l’évènement A : «  le joueur obtient 8 » est : $\mathbf{P(A) = 	\dfrac{4}{36}}$.
	\end{enumerate}
	\item 
	\begin{enumerate}
		\item La probabilité de l’évènement « il obtient une patte à sa fourmi dès son premier lancer » correspond à la probabilité de l’évènement B « il obtient 6 à son lancer ».
		\begin{center}
\noindent\begin{tabularx}{\linewidth}{|c|*{6}{>{\centering\arraybackslash}X|}}
\hline 
\diagbox{Dé 2}{Dé 1} & 1 & 1 & 2 & 3 & 4 & 5 \\ 
\hline 
1 & 2 & 2 & 3 & 4 & 5 & \cellcolor{gray!30}{6} \\ 
\hline 
2 & 3 & 3 & 4 & 5 & \cellcolor{gray!30}{6} & 7 \\ 
\hline 
3 & 4 & 4 & 5 & \cellcolor{gray!30}{6} & 7 & 8 \\ 
\hline 
4 & 5 & 5 & \cellcolor{gray!30}{6} & 7 & 8 & 9 \\ 
\hline 
5 & \cellcolor{gray!30}{6} & \cellcolor{gray!30}{6} & 7 & 8 & 9 & 10 \\ 
\hline 
5 & \cellcolor{gray!30}{6} & \cellcolor{gray!30}{6} & 7 & 8 & 9 & 10 \\ 
\hline 
\end{tabularx} 
		\end{center}
		$ \mathbf{P(B) = \dfrac{8}{36}=\dfrac{2}{9}}$
		\item La probabilité de l’évènement C « il obtient 2 pattes à sa fourmi en deux lancers » est : $\mathbf{P(C) = \dfrac{2}{9}\times \dfrac{2}{9} = \dfrac{4}{81}}$.
	\end{enumerate}
	\item 
	\begin{enumerate}
		\item Eden a choisi le nombre qui a la plus grande probabilité d’être obtenu (voir tableau), elle a donc plus de chance de gagner la partie. 
		\item Néanmoins il s’agit de valeurs théoriques, elle a plus de chance de gagner ce qui ne veut pas dire qu’elle va forcément gagner.
	\end{enumerate}
\end{enumerate}