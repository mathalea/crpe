\begin{enumerate}
\item Je calcule la longueur totale du parcours :

$AB + BC + CD + DA = 960~\text{m} + 1,05~\text{km} + 780~\text{m} + 660~\text{m} = 960~\text{m} + 1050~\text{m} + 780~\text{m} + 660~\text{m} = 3~450~\text{m}$

La longueur totale du parcours vaut donc : $\mathbf{3~450~\text{\textbf{m}}}$
\item 
	\begin{enumerate}
		\item La distance parcourue par Léo est : $ d = 2 \times 3~450+\dfrac{1}{3}\times 3~450 = 8~050~\text{m}$
		\item Léo parcourt donc $8~050~\text{m} = 8,05~\text{km}$ en $48~\text{min}$ soit en $\dfrac{48}{60}~\text{h} = 0,8~\text{h}$.
		
		Sa vitesse moyenne est donc de $v=\dfrac{8,05~\text{km}}{0,8~\text{h}=10,0625~\text{km/h}}$
		\item S'il court à cette vitesse moyenne pendant 15 km : $t=\dfrac{15~\text{km}}{10,0624~\text{km/h}} \simeq 1,49~\text{h}$ au centième près.
		
		Léo mettra moins d’une heure et demie pour parcourir 15 km.
	\end{enumerate}
\item Tara parcourt $2,01~\text{km}$ $(960~\text{m} + 1,05~\text{km})$ à une vitesse moyenne de $10~\text{km/h}$. $T_{Tara}=\dfrac{2,01~\text{km}}{10~\text{km/h}}=0,201~\text{h}$

Kévin parcourt $1,44~\text{km}$ $(780~\text{m} + 660~\text{m})$ à une vitesse moyenne de $6~\text{km/h}$.
$T_{Kévin}=\dfrac{1,44~\text{km}}{6~\text{km/h}}=0,24~\text{h}$

Tara et Kévin parcourt donc $3,45~\text{km}$ en $0,441~\text{h}$ soit une vitesse moyenne
$v=\dfrac{3,45~\text{km}}{0,441~\text{h}}=7,82~\text{km/h}$
\item 
	\begin{enumerate}
		\item En utilisant cette échelle on en déduit que les longueurs réelles sont multipliées par $\dfrac{1}{20~000}$
		
		On peut ainsi calculer : $AB_{plan}= \dfrac{1}{20~000}\times 960~\text{m}=0,048~\text{m}=4,8~\text{cm}$
		
		On peut également utiliser le tableau de proportionnalité suivant. $1$ cm sur le plan représente $20~000$ cm en réalité soit $200$ m.
		
		\begin{center}
\noindent\begin{tabularx}{\linewidth}{|c|*{6}{>{\centering\arraybackslash}X|}}
\hline 
 &  & $AB$ & $BC$ & $CD$ & $DA$ & $BD$\\ 
\hline 
Longueur sur le plan & 1 & 4,8 & 5,25 & 3,9 & 3,3 & 5,25\\ 
\hline 
Longueur réelle en m & 200 & 960 & 1 050 & 780 & 660 & 1 050\\ 
\hline 
\end{tabularx} 
\end{center}
		\item Amina a roulé pendant 25 minutes à 11,5 km/h elle a parcouru une distance égale à :
		
		$d=\dfrac{25}{60}~\text{h} \times 11,5~\text{km/h}=\dfrac{115}{24}~text{km} \simeq 4~792~\text{m}$
		
		\textbf{Elle parcourt donc un tour complet (3 450 m) + la longueur AB (960m) elle se trouve à 382 m du point B soit sur le plan à 1,91 cm.}

\begin{center}
\begin{tikzpicture}[x=1cm,y=1cm]
\draw (-6.02,3.89)-- (-1.22,3.89);
\draw (-6.02,3.89)-- (-5.35671875,0.657345056552055);
\draw (-5.35671875,0.657345056552055)-- (-1.22,3.89);
\draw (-1.22,3.89)-- (-1.9857137866622296,-1.3038600671288199);
\draw (-1.9857137866622296,-1.3038600671288199)-- (-5.35671875,0.657345056552055);
\begin{scriptsize}
\draw (-5.86,4.1) node {$A$};
\draw (-1.06,4.1) node {$B$};
\draw (-5.64,0.6) node {$D$};
\draw (-1.9,-1.5) node {$C$};
\draw [fill=black] (-1.5,2) circle (1pt);
\draw (-1.2,1.82) node {$S$};
\draw (-3.38,3.6) node {$4,8$};
\draw (-6,2.44) node {$3,3$};
\draw (-2.8,2.2) node {$5,25$};
\draw (-1.9,1.58) node {$5,25$};
\draw (-3.74,-0.58) node {$3,9$};
\draw (-0.9,3) node {$1,91$};
\end{scriptsize}
\end{tikzpicture}
\end{center}
	\end{enumerate}
\end{enumerate}