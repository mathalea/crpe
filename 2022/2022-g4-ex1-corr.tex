\textbf{On considère que pour l'ensemble de l'exercice, on est dans une situation d'équiprobabilité. Chaque carte a la même probabilité d'être tirée.}

\begin{enumerate}
    \item 
    \begin{enumerate}
        \item La probabilité que Déborah tire une carte bleue est donnée par le quotient :
        
        \medskip
        $\dfrac{\text{nombre de cartes bleues}}{\text{nombre de cartes au total}}=\dfrac{20}{80}=\dfrac{1}{4}$
        \item La probabilité que Déborah tire une carte portant le numéro 2 est donnée par le quotient :
        
        \medskip
        $\dfrac{\text{nombre de cartes portant le numéro 2}}{\text{nombre de cartes au total}}=\dfrac{8}{80}=\dfrac{1}{10}$
        \item La probabilité que Déborah tire une carte bleue portant le numéro 2 est : $\dfrac{2}{80}=\dfrac{1}{40}$
        \item On cherche la probabilité que Déborah tire une carte bleue ou portant le numéro 2. On sait qu'il y a 20 cartes bleues. Il faut ajouter le nombre de cartes portant le numéro 2 et non bleues, il y en a 6. On a donc 26 cartes bleues ou portant le numéro 2. La probabilité que Déborah tire une carte bleue ou portant le numéro 2 est : $\dfrac{26}{80}=\dfrac{13}{40}$
    \end{enumerate}
    \item On cherche le nombre de cartes Joker à ajouter tel que la probabilité de piocher une carte Joker soit de $\dfrac{1}{6}$. Appelons $n$ le nombre de cartes Joker, on cherche $n$ tel que : \\$\dfrac{n}{n+80}=\dfrac{1}{6}$ soit $n\times6=(n+80)\times1$ ; $6n=n+80$ ; $5n=80$ ; $n=\dfrac{80}{5}=16$.\\Il faut ajouter 16 cartes Joker.
    
\end{enumerate}