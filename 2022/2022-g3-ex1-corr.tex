\textbf{Question 1 : réponse B}
\medskip

$\text{Volume = aire de la base}\times \text{hauteur} = \pi \times r^2 \times \text{hauteur} = \pi \times 4^2 \times 6~\text{cm\up{3}}=96\pi~\text{cm\up{3}}$ 

\medskip
\textbf{Question 2 : réponse C}
\medskip

Le $1$\up{er} juin Nicolas partage avec $3$ personnes ;

Le deuxième jour (2 juin) chaque personne partage avec 3 nouvelles personnes soit $3\times 3$;

Le troisième jour chaque personne partage avec $3$ nouvelles personnes soit $3 \times (3 \times 3)$.

\ldots

Le 10 juin, il y aura donc $3^{10}=59~049$ personnes qui apprendront la rumeur. 

\medskip
\textbf{Question 3 : réponse A}
\medskip

Augmenter de $10$\% revient à multiplier par $1,1$.

Baisser de $10$\% revient à multiplier par $0,9$. 

Les deux évolutions successives reviennent à multiplier par $1,1 \times 0,9 = 0,99$ soit une baisse de $1$\%.

\medskip
\textbf{Question 4 : réponse C}
\medskip

$\dfrac{4}{25} = \dfrac{16}{100}$, est donc un nombre décimal mais n’est pas un nombre entier. 

\medskip
\textbf{Question 5 : réponse D}
\medskip

Le quart de $\dfrac{4}{12}$ est : $\dfrac{1}{4}\times \dfrac{4}{12} = \dfrac{4}{48}$.

\medskip
\textbf{Question 6 : réponse A}
\medskip

$\dfrac{2}{3}\times 5 + 5\times \dfrac{1}{3} = 5\times \left(\dfrac{2}{3} + \dfrac{1}{3}\right) = 5\times 1 = 5$

\medskip
\textbf{Question 7 : réponse A}
\medskip

Pour cela il faut déjà calculer la longueur $BC$.

On sait que le triangle $ABC$ est rectangle en $B$ donc d’après le théorème de Pythagore :
\begin{spacing}{2}
$AC^2 = AB^2 + BC^2$

$BC^2 = AC^2 - AB^2 = 10^2 - 8^2 = 100 - 64 = 36$ donc $BC = 6$ cm

Donc $\text{Aire}_{ABC} = \dfrac{AB \times BC}{2}=\dfrac{8\times 6}{2}~\text{cm\up{2}}=24~\text{cm\up{2}}$
\end{spacing}
