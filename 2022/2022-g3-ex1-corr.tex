Pour chacune des questions, une seule des quatre réponses proposées est exacte. Donner la
bonne réponse en la justifiant.

\textit{Une réponse erronée n’enlève pas de point. Une réponse non justifiée ne rapporte pas de
point.}

\begin{center}

	\renewcommand{\arraystretch}{3}
	\noindent\begin{tabularx}{0.9\linewidth}{|>{\hsize=7cm}X|*{4}{>{\centering \arraybackslash}X|}}
	\hline 
	\textbf{Questions :} & \textbf{A} & \textbf{B} & \textbf{C} & \textbf{D} \\ 
	\hline 		
		\textbf{1.} Quel est le volume d'un cylindre
		d'une hauteur de 6 cm et de base un
		disque d’un diamètre de 8 cm ?
		On rappelle que le volume d'un
		cylindre se calcule avec la formule
		suivante : $\text{aire de la base} \times \text{hauteur}$
		& $48\pi~\text{cm}^3$ & $96\pi~\text{cm}^3$ & $144\pi~\text{cm}^3$ & $384\pi~\text{cm}^3$ \\ 
	\hline 	
	\textbf{2.} Le $1^{er}$ juin, Nicolas lance une rumeur
	en la partageant avec trois	personnes. Chaque jour, une
	personne prévenue la veille prévient trois nouvelles personnes qui ne sont
	pas encore informées. Combien de personnes apprennent la rumeur le 10 juin ?	
	& $30$ & $1~000$ & $59~049$ & $177~147$ \\ 
	\hline 	
	\textbf{3.}	Le prix d’un article subit une hausse
	de \par 10 \% suivie d’une baisse de 10 \%
	quelques semaines plus tard. Au
	final :	
 	&
	le prix de	l’article a	baissé de \par 1 \%.
	&
	l'article a retrouvé son prix initial.
	&
	le prix de l'article a augmenté de 1 \%.
	&
	le prix de l'article a augmenté de 5 \%.
	\\ 
	\hline 	
	\textbf{4.}	$\dfrac{4}{25}$ est \ldots
	&
	un nombre réel mais n'est pas un nombre rationnel.
	&
	un nombre rationnel mais n'est pas un nombre décimal.
	&
	un nombre décimal mais n'est pas un nombre entier.
	&
	un nombre entier.
	\\ 
	\hline 
	\textbf{5.} Le quart de $\dfrac{4}{12}$ est \ldots	
	& $\dfrac{1}{3}$ & $\dfrac{4}{3}$ & $\dfrac{16}{48}$ & $\dfrac{4}{48}$ \\ 
	\hline 	
	\textbf{6.} $\dfrac{2}{3}\times 5  5 \times \dfrac{1}{3}$ est égal à	
	& $5$ & $\dfrac{20}{9}$ & $\dfrac{15}{15}$ & $\dfrac{20}{90}$ \\ 
	\hline 	
	\textbf{7.} Le triangle $ABC$ est rectangle en $B$.	
	De plus, $AB = 8$ cm et $AC = 10$ cm.
	L’aire du triangle $ABC$ est \ldots	
	& $24~\text{cm}^2$ & $40~\text{cm}^2$ & $48~\text{cm}^2$ & $80~\text{cm}^2$ \\ 
	\hline 
	\end{tabularx}	
	\renewcommand{\arraystretch}{1} 	
\end{center}