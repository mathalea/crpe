
On peut résumer la situation avec le tableau suivant : 

\begin{center}
\medskip
\begin{tabular}{*{7}{|c}|}
	\hline
	\backslashbox{Dé 2}{Dé 1} & 0 & 0 & 1 & 1 & 2 & 2 \\
	 \hline
	 0 & 0,0 & 0,0 & 1,0 & 1,0 & 2,0 & 2,0 \\
	 \hline
	 0 & 0,0 & 0,0 & 1,0 & 1,0 & 2,0 & 2,0 \\
	 \hline
	 1 & 0,1 & 0,1 & 1,1 & 1,1 & 2,1 & 2,1 \\
	 \hline	
	 1 & 0,1 & 0,1 & 1,1 & 1,1 & 2,1 & 2,1 \\
	 \hline	
	 2 & 0,2 & 0,2 & 1,2 & 1,2 & 2,2 & 2,2 \\
	 \hline	
	 2 & 0,2 & 0,2 & 1,2 & 1,2 & 2,2 & 2,2 \\
	 \hline	
\end{tabular}
	
\end{center}

\bigskip
\begin{enumerate}
	\item 
	\begin{enumerate}
		\item Les nombres qu'on peut obtenir sont : 0,0 ; 1,2 ; 1,0 ; 1,1 ;  1,2 ;  2,0 ; 2,1 ; 2,2.
		\item Toutes les issues présentes dans le tableau ont la même probabilité d'être obtenue.\\
		La probabilité de l'évènement $A$ \og{} obtenir 1,2 \fg{} est $P(A)=\dfrac{4}{36}=\dfrac{1}{9}$.
		
		\item Soit l'évènement $B$ \og{} obtenir un nombre strictement inférieur à 1\fg{}.
		
		\medskip
		\begin{tabular}{*{7}{|c}|}
	\hline
	\backslashbox{Dé 2}{Dé 1} & 0 & 0 & 1 & 1 & 2 & 2 \\
	 \hline
	 0 & \bfseries 0,0 & \bfseries 0,0 & 1,0 & 1,0 & 2,0 & 2,0 \\
	 \hline
	 0 & \bfseries 0,0 & \bfseries 0,0 & 1,0 & 1,0 & 2,0 & 2,0 \\
	 \hline
	 1 & \bfseries 0,1 & \bfseries 0,1 & 1,1 & 1,1 & 2,1 & 2,1 \\
	 \hline	
	 1 & \bfseries 0,1 & \bfseries 0,1 & 1,1 & 1,1 & 2,1 & 2,1 \\
	 \hline	
	 2 & \bfseries 0,2 & \bfseries 0,2 & 1,2 & 1,2 & 2,2 & 2,2 \\
	 \hline	
	 2 & \bfseries 0,2 & \bfseries 0,2 & 1,2 & 1,2 & 2,2 & 2,2 \\
	 \hline	
\end{tabular}

	\medskip On dénombre dans le tableau 12 issues favorables donc $P(B)=\dfrac{12}{36}=\dfrac{1}{3}$.
	
	\item Soit $C$ l'évènement \og{} obtenir un nombre entier \fg{} : 
	
	\medskip
\begin{tabular}{*{7}{|c}|}
	\hline
	\backslashbox{Dé 2}{Dé 1} & 0 & 0 & 1 & 1 & 2 & 2 \\
	 \hline
	 0 &\bfseries 0,0 & \bfseries 0,0 & \bfseries 1,0 & \bfseries 1,0 & \bfseries 2,0 & \bfseries 2,0 \\
	 \hline
	 0 & \bfseries 0,0 & \bfseries 0,0 & \bfseries 1,0 & \bfseries 1,0 & \bfseries 2,0 & \bfseries 2,0 \\
	 \hline
	 1 & 0,1 & 0,1 & 1,1 & 1,1 & 2,1 & 2,1 \\
	 \hline	
	 1 & 0,1 & 0,1 & 1,1 & 1,1 & 2,1 & 2,1 \\
	 \hline	
	 2 & 0,2 & 0,2 & 1,2 & 1,2 & 2,2 & 2,2 \\
	 \hline	
	 2 & 0,2 & 0,2 & 1,2 & 1,2 & 2,2 & 2,2 \\
	 \hline	
\end{tabular}

	\medskip On dénombre dans le tableau 12 issues favorables donc $P(C)=\dfrac{12}{36}=\dfrac{1}{3}$.
	
	\item Soit $D$ l'évènement \og{} obtenir un nombre décimal \fg{}. $D$ est un évènement certain (car les nombres entiers sont aussi des nombres décimaux) donc $P(D)=1$.
	\end{enumerate}
	
	\item 
	\begin{enumerate}
		\item Soit $E$ l'évènement \og{} le dé tombe sur la zone $Z_2$.\fg{}. $P(E)=\dfrac{12}{36}=\dfrac{1}{3}$.
		
		\item Soit $F$ l'évènement \og{} obtenir 1 avec le dé \fg{}. $P(F)=\dfrac{2}{6}=\dfrac{1}{3}$ donc $P(\og{} F~\text{et }E\fg)=\dfrac{1}{3}\times\dfrac{1}{3}=\dfrac{1}{9}$.
		
		\item Soit $G$ l'évènement \og{} obtenir un nombre pair avec le dé \fg{}. $P(G)=\dfrac{4}{6}=\dfrac{2}{3}$ donc $P(\og{} G~\text{et }E\fg)=\dfrac{2}{3}\times\dfrac{1}{3}=\dfrac{2}{9}$.
		
		 
	\end{enumerate}

\end{enumerate}