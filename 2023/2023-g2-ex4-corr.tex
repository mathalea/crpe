\begin{enumerate}
    \item D'après l'énoncé, il y a équiprobabilité pour le choix d'une case.
       \begin{enumerate}
          \item $\mathcal{P}_{1a} =\dfrac{\text{nombre de cas favorables}}{\text{nombre de cas possibles}} =\dfrac{2+2\times3+4}{8\times8} =\dfrac{12}{64} =\dfrac{3}{16}$. \par
             \con{La probabilité que Jeanne touche un bateau est de $\dfrac{3}{16}$}.
          \item $\mathcal{P}_{1b} =1-\dfrac{3}{16} =\dfrac{13}{16}$. \par \smallskip
             \con{La probabilité que Jeanne ne touche aucun bateau est de $\dfrac{13}{16}$}.
          \item $\dfrac{1}{16} =\dfrac{1\times4}{16\times4} =\dfrac{4}{64}$. \par \smallskip
             \con{Un bateau ayant une chance de un sur seize d'être touché est installé sur quatre cases}.
          \item $\mathcal{P}_{1d} =\dfrac{\text{nombre de cas favorables}}{\text{nombre de cas possibles}} =\dfrac{1+3}{8} =\dfrac48 =\dfrac12$. \par \smallskip
          \con{La probabilité que Jeanne touche un bateau sachant qu'elle choisit la colonne B est de un demi.}
       \end{enumerate}
    \item Pour couler le bateau au coup suivant, Jeanne doit désigner la case F1. Or, elle a trois possibilités de choisir une case adjacente à la case E1 : il s'agit des cases D1, E2 et F1. \par
       $\mathcal{P}_{2} =\dfrac{\text{nombre de cas favorables}}{\text{nombre de cas possibles}} =\dfrac13$. \par
       \con{La probabilité que Jeanne coule le bateau au deuxième essai est de un tiers.}
    \item Les deux premiers essais ont été infructueux, il reste toujours les quatre bateaux à couler, composés au total de 12 cases sur un total de $(64-2)$ cases = 62 cases disponibles. D'où la probabilité : \par
       $\mathcal{P}_{3} =\dfrac{\text{nombre de cas favorables}}{\text{nombre de cas possibles}} =\dfrac{12}{62} =\dfrac{6}{31}$. \par \vskip-1mm
       \con{La probabilité que Jeanne touche un bateau au troisième essai est de $\dfrac{6}{31}$.}
 \end{enumerate}