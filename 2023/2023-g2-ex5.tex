Pour choisir une unité de température, les physiciens se sont heurtés à l’absence de << température zéro >> (le zéro absolu n’était pas connu à l’époque). Deux systèmes principaux ont été créés et restent utilisés : le degré Celsius (°C) et le degré Fahrenheit (°F). \par
Voici ci-dessous une formule permettant de passer de la mesure d’une température en degré Fahrenheit (notée $F$) vers la mesure de la même température en degré Celsius (notée $C$).
\begin{center}
   $C =(F-32)\times\dfrac59$
\end{center}
\begin{enumerate}
   \setlength{\itemsep}{-1mm}
   \item En utilisant cette formule, convertir 95°F en degré Celsius.
   \item En utilisant cette formule, convertir 5°C en degré Fahrenheit.
   \item Existe-t-il des températures pour lesquelles la mesure en degrés Celsius est égale à la mesure en degrés Fahrenheit ? Donner toutes les réponse possibles en justifiant.
\end{enumerate}