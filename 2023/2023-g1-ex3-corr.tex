\begin{enumerate}
    \item 
       \begin{enumerate}
          \item Les coordonnées du point de départ sont \con{$(-75;50)$.}
          \item \con{Cinq triangles} sont dessinés par le script.
          \item Les triangles dessinés sont des \con{triangles équilatéraux.}
          \item La longueur d'un côté du deuxième triangle est de \con{80 pas.}
       \end{enumerate}
    \item Dessin à l'échelle 1 cm pour 20 pas.
       \begin{center}
          \begin{pspicture}(0,0)(15,4.5)
             \psset{linecolor=violet}
             \pspolygon(0,0)(5,0)(5;60)
             \rput(5,0){\pspolygon(0,0)(4,0)(4;60)}
             \rput(9,0){\pspolygon(0,0)(3,0)(3;60)}
             \rput(12,0){\pspolygon(0,0)(2,0)(2;60)}
             \rput(14,0){\pspolygon(0,0)(1,0)(1;60)}
          \end{pspicture}
       \end{center}
    \item Pour un hexagone régulier, il faudrait tout d'abord modifier le nombre de côtés et donc remplacer le bloc 3 du bloc triangle par \con{répéter 6 fois}. \par
       D'autre part, l'angle de rotation du bloc 5 devrait être modifié en \con{tourner à gauche de 60 degrés}.
 \end{enumerate}