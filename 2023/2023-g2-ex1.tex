Une directrice d’école primaire souhaite inscrire les élèves de l’école à une course solidaire d’action contre la faim afin de les sensibiliser à la sous-nutrition dans le monde. \par
   Il s’agit pour chaque élève de faire le plus de tours possible d’un parcours prédéfini. Pour chaque tour effectué, l’élève récolte une somme d’argent fixe qui sera versée à l’association caritative. \par
   La directrice décide de faire courir les élèves dans la cour de l’école, le long d’un parcours schématisé ci-dessous. \par \bigskip
   \begin{minipage}{8.5cm}
      Une partie du parcours est constituée d’un demi-cercle de diamètre $[CD]$ et les longueurs sont données en mètre. \par
      Les points $A$, $B$, et $D$ sont alignés et le quadrilatère $AFED$ est un rectangle. \par
      Les élèves partent du point $A$ et se déplacent dans le sens inverse des aiguilles d’une montre. \par
      On a : $AB =\Lg[m]{50}$ ; $BC =\Lg[m]{40}$ ; $EF =\Lg[m]{80}$ et $FA =\Lg[m]{30}$.
   \end{minipage}
   \begin{center}
      \begin{pspicture}(-5,-0.5)(9,4)
         \pstGeonode[CurveType=polygon,PosAngle={-135,-45,-45,135,-45,135},PointSymbol=none]{F}(8,0){E}(8,3){D}(5,7){C}(5,3){B}(0,3){A}
         \pstLineAB[linecolor=orange!15,linewidth=0.5mm,nodesep=0.2mm]{C}{D}
         \psarc(6.5,5){2.5}{-53}{127}
         \psdot(0,3)
         \pstLabelAB[offset=-3mm]{F}{E}{80 m}
         \pstLabelAB[offset=3mm]{F}{A}{30 m}
         \pstLabelAB[offset=3mm]{A}{B}{50 m}
         \pstLabelAB[offset=3mm]{B}{C}{40 m}
         \pstRightAngle[fillstyle=solid,fillcolor=gray!25]{A}{B}{C}
         \psline{->}(-0.5,3)(-0.5,2)
         \rput(-1.1,2.5){\small Départ}
      \end{pspicture}
   \end{center}
   \begin{enumerate}
      \setlength{\itemsep}{-1mm}
      \item Calculer la longueur du segment $[CD]$.
      \item Montrer que la longueur du parcours, arrondie au mètre, est \Lg[m]{309}. \newline
         On utilisera cette valeur dans la suite de l’exercice.
      \item Construire un plan du parcours à l’échelle 1/800. 
      \item Killian a effectué un tour complet en 3 minutes. \newline
         À quelle vitesse moyenne Killian a-t-il couru ? On donnera le résultat en mètre par seconde arrondi au centième, puis en kilomètre par heure, arrondi au dixième. 
      \item On suppose que Sophia court à une vitesse constante de \Vitesse{7}.
         \begin{enumerate}
            \setlength{\itemsep}{-1mm}
            \item Combien de tours complets pourrait-elle effectuer à cette vitesse en 18 minutes ?
            \item On désigne par $S$ le point du parcours où Sophia se trouve au bout de 18 minutes de course. \newline
               Placer le point $S$ sur le plan réalisé à la question 3.
         \end{enumerate}
      \item L’école est composée de 325 élèves. Le tableau ci-dessous indique le nombre de tours complets effectués par les élèves.
         \begin{center}
            \hautab{1.2}
            \begin{tabular}{|p{3cm}|*{7}{C{1}|}}
               \hline
               Nombre de tours & 2 & 3 & 4 & 5 & 6 & 7 & 8 \\
               \hline
               Nombre d'élèves & 52 & 52 & 78 & 65 & 39 & 26 & 13 \\
               \hline
            \end{tabular}
         \end{center}
         \begin{enumerate}
            \setlength{\itemsep}{-1mm}
            \item Quel est le nombre moyen de tours complets effectués ?
            \item Quelle est l’étendue de cette série statistique ?
            \item Déterminer la médiane de cette série statistique.
            \item Interpréter le résultat de la question c.
            \item Déterminer le premier et le troisième quartile de cette série.
            \item Quel pourcentage d’élèves a réussi à faire au moins 4 tours ?
         \end{enumerate}
   \end{enumerate}