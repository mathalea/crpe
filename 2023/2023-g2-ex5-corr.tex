\begin{enumerate}
    \item Dans cette question, on nous donne $F =95$ (en °F) et on cherche $C$ (en °C). \par \vskip-3mm
       $$C =(95-32)\times\dfrac59  =63\times\dfrac59 =35$$
       \con{95°F correspondent à une température de 35°C.}
    \item Dans cette question, on nous donne $C =5$ (en °C) et on cherche $F$ (en °F). \vskip-7mm
       \begin{align*}
          5 &=(F-32)\times\dfrac59 \\
          5\times\dfrac95 &=F-32 \\
          9+32 &= F \\
          41 &=F
       \end{align*}
       \con{5°C correspondent à une température de 41°F.}
    \item Les deux températures sont égales si, et seulement si, $C =F$. \vskip-7mm
       \begin{align*}
          F &=(F-32)\times\dfrac59 \\
          F\times\dfrac95 &=F-32 \\
          \dfrac95F-F &=-32 \\
          \dfrac45F &=-32 \\
          F &=\dfrac54\times(-32) \\
          F &=-40
       \end{align*}
       \con{Il existe une unique température pour laquelle les deux mesures sont égales, il s'agit de $-$40°C = $-$40°F}
 \end{enumerate}