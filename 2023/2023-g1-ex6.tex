Dans une école élémentaire de 150 élèves, 80 sont des filles. Le directeur veut mettre en place un << orchestre à l’école >>. Il réalise une enquête auprès des familles de l’école afin de connaître les élèves qui pratiquent déjà un instrument de musique. \newline
   À l’issue de l’enquête, il apparaît que 24\,\% des élèves sont musiciens. Parmi ces élèves, 16 sont des garçons.
   \begin{enumerate}
      \setlength{\itemsep}{-1mm}
      \item Reproduire et compléter le tableau suivant.
         \begin{center}
            {\hautab{1.2}
            \begin{tabular}{|C{3}|C{3}|C{3}|C{3}|}
               \hline
               & Nombre d'élèves musiciens & Nombre d'élèves non-musiciens & Total \\
               \hline
               Nombre de filles & & & \\
               \hline
               Nombre de garçons & & & \\
               \hline
               Total & & & 150 \\
               \hline
            \end{tabular}}
         \end{center}
      \item Dans cette question, on écrira les résultats sous forme de fractions irréductibles. \newline
         On interroge un élève au hasard.
         \begin{enumerate}
            \item Quelle est la probabilité pour que ce soit un garçon ?
            \item Quelle est la probabilité que ce soit une fille musicienne ?
            \item Quelle est la probabilité que ce soit un élève non-musicien ?
         \end{enumerate}
      \item L’élève interrogé est un garçon. Quelle est la probabilité qu’il soit musicien ?
      \item 30\,\% des filles musiciennes jouent d’un instrument à vent. \newline
         Quel pourcentage cela représente-t-il par rapport à l’effectif total de l’école ?   
   \end{enumerate}