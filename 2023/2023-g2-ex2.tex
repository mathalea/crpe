Un rectangle est défini dans le dictionnaire de la façon suivante : \par
   << {\it Un rectangle est un quadrilatère dont les quatre angles sont droits.} >>
   \begin{enumerate}
      \setlength{\itemsep}{-1mm}
      \item Un quadrilatère qui possède deux angles droits est-il un rectangle ? Justifier.
      \item Dans une classe de CE2, une enseignante demande à ses élèves de compléter la phrase suivante : \newline
         << {\it Un rectangle est un quadrilatère dont \dots} >>. Voici deux réponses proposées : \newline
         Élève A : << {\it Un rectangle est un quadrilatère dont les côtés opposés sont de même longueur} >>. \newline
         Élève B : << {\it Un rectangle est un quadrilatère dont les diagonales sont de même longueur} >>.
         \begin{enumerate}
            \setlength{\itemsep}{-1mm}
            \item Préciser en quoi la réponse de l’élève A ne pourrait pas être admise comme définition mathématique du rectangle. 
            \item Préciser en quoi la réponse de l’élève B ne pourrait pas être admise comme définition mathématique du rectangle.
         \end{enumerate}
      \item Qu’elle est la nature d’un rectangle dont les diagonales sont perpendiculaires ?
      \item En s’appuyant sur le codage du quadrilatère ci-après dessiné à main levée, préciser la nature du quadrilatère en question en justifiant la réponse.
         \begin{center}
            \begin{pspicture}(0,-0.5)(4,2.5)
               \pslineByHand[VarStepEpsilon=1,varsteptol=0.4](0,0)(4,0)(4,2.5)(0,2.5)(0,0)
               \pstGeonode[PosAngle={-135,-45,45,135},PointSymbol=none]{D}(4,0){C}(4,2.5){B}(0,2.5){A}
               \rput(2,0){$\times$}
               \rput(2,2.5){$\times$}
               \rput(0,1.25){$\times$}
               \rput(4,1.25){$\times$}
            \end{pspicture}
         \end{center}
   \end{enumerate}