Dans une classe de Grande Section, l’enseignant propose à ses élèves le jeu suivant dans lequel il s’agit d’être le premier à avoir \underline{exactement} 15 jetons (source : {\it Découvrir les maths GS - Éditions Hatier}). \newline
   Chaque élève lance deux dés bien équilibrés, identiques, à 6 faces numérotées de 1 à 6. Il considère les deux nombres indiqués sur les faces supérieures de chacun des dés. \newline
   Lorsque les deux dés indiquent le même nombre, l’élève prend autant de jetons que l’indique l’un des deux dés. Sinon, il prend autant de jetons que le plus grand des deux nombres ou le double de jetons du plus petit. \newline
   Après avoir lancé les dés, un élève a la possibilité de passer son tour. Dans ce cas, il ne prend aucun jeton.
   \begin{enumerate}
      \setlength{\itemsep}{-1mm}
      \item Un élève lance les deux dés ; il obtient un 3 et un 2. Combien de jetons peut-il prendre ? \newline
         Donner tous les cas possibles.
      \item Dresser la liste des tirages permettant d’obtenir 3 jetons.
      \item Un élève lance les deux dés.
         \begin{enumerate}
            \setlength{\itemsep}{-1mm}
            \item Montrer que la probabilité de l’événement << les nombres obtenus sont un 3 et un 2 >> est de $\dfrac{1}{18}$.
            \item Quelle est la probabilité de l’événement << au moins un des nombres obtenus est 3 >> ?
            \item Quelle est la probabilité de l’événement << les nombres obtenus permettent de prendre 4 jetons >> ?
         \end{enumerate}
      \item Après un nouveau lancer des deux dés, un élève a pris 3 jetons. Au lancer suivant, la probabilité qu’il prenne de nouveau 3 jetons augmente-t-elle, reste-t-elle identique ou diminue-t-elle par rapport à la probabilité d’avoir pris 3 jetons au tirage précédent ? Justifier.
      \item En fin de partie, un élève possède 12 jetons. Lors de son lancer de dés, il obtient un 1 et un 4. Pourquoi est-il préférable pour lui de passer son tour ?
   \end{enumerate}