
\begin{enumerate}
    \item Tableau récapitulatif :
       \begin{center}
          {\hautab{1.2}
          \begin{tabular}{|C{3}|C{3}|C{3}|C{3}|}
             \hline
             & Nombre d'élèves musiciens & Nombre d'élèves non-musiciens & Total \\
             \hline
             Nombre de filles & \textcolor{violet}{20} & \textcolor{violet}{60} & \textcolor{violet}{80} \\
             \hline
             Nombre de garçons & \textcolor{violet}{16} & \textcolor{violet}{54} & \textcolor{violet}{70} \\
             \hline
             Total & \textcolor{violet}{36} & \textcolor{violet}{114} & 150 \\
             \hline
          \end{tabular}}
       \end{center}
    \item On interroge un élève au hasard. Il y a donc équiprobabilité entre tous les élèves.
       \begin{enumerate}
          \item $\mathcal{P}_{2a} =\dfrac{\text{nombre de garçons}}{\text{nombre total d'élèves}} =\dfrac{70}{150} =\dfrac{7}{15}$. \par
             \con{La probabilité que l'élève choisi soit un garçon est de $\dfrac{7}{15}$.}
          \item $\mathcal{P}_{2b} =\dfrac{\text{nombre de filles musiciennes}}{\text{nombre total d'élèves}} =\dfrac{20}{150} =\dfrac{2}{15}$. \par
             \con{La probabilité que l'élève choisi soit une fille musicienne est de $\dfrac{2}{15}$.}
          \item $\mathcal{P}_{2c} =\dfrac{\text{nombre d'élèves non-musiciens}}{\text{nombre total d'élèves}} =\dfrac{114}{150} =\dfrac{19}{25}$. \par
             \con{La probabilité que l'élève choisi soit un élève non musicien est de $\dfrac{19}{25}$.}
       \end{enumerate}
    \item $\mathcal{P}_3 =\dfrac{\text{nombre de garçons musiciens}}{\text{nombre de garçons}} =\dfrac{16}{70} =\dfrac{8}{35}$. \par
       \con{La probabilité que l'élève choisi soit musicien, sachant que c'est un garçon est de $\dfrac{8}{35}$.}
    \item 30\,\% des 80 filles représentent $\dfrac{30}{100}\times20\text{ filles} =6\text{ filles}$. \par
       Il y a donc 6 filles qui jouent un instrument à vent sur un total de 150 élèves. \par
       Cela représente un pourcentage de $\dfrac{6}{150}\times100 =4\,\%$. \par
       \con{4\,\% des élèves sont des filles jouant un instrument à vent.}
 \end{enumerate}