\begin{enumerate}
    \item Dans toute cette question, les longueurs sont exprimées en mètre.
       \begin{enumerate}
          \item \Pythagore[Unite=m,Exact]{BAC}{300}{400}{}
             \con{La longueur $BC$ est égale à \Lg[m]{500}.}
          \item \Thales[Figure,Angle=130,Echelle=12mm,IntroCalculs=false,Unite=m]{CABFI}{CF}{350}{FI}{400}{500}{300}
             \con{La longueur $IF$ vaut \Lg[m]{210} et la longueur $CF$ est égale à \Lg{280}.}
          \item \Thales[Figurecroisee,Angle=100,Echelle=12mm,Droites,IntroCalculs=false,Unite=m,ChoixCalcul=3]{CEDAB}{CE}{1250}{ED}{400}{500}{300}
             \con{La longueur $ED$ vaut \Lg[m]{750}.}
          \item On note $\ell$ la longueur du parcours $ABIFCDE$.
             \begin{align*}
                \ell &=AB+BI+IF+FC+CD+DE \\ 
                &=\Lg[m]{300}+(\Lg[m]{500}-\Lg[m]{350})+\Lg[m]{210}+\Lg[m]{280}+\Lg[m]{1250}+\Lg[m]{750} \\
                &=\Lg[m]{2940}
             \end{align*}
             \con{Le parcours $ABIFCDE$ mesure \Lg[m]{2940}, ou \Lg[km]{2,94}.}
       \end{enumerate}
    \item Graphique avec les traces de recherche.
       \begin{center}
          \psset{xunit=1.55,yunit=1.2}
          \begin{pspicture}(-0.5,-1.2)(9,5.5)
             \psgrid[subgriddiv=5, gridlabels=0pt,gridcolor=gray,subgridcolor=gray!80](0,0)(9.2,5.2)
             \psaxes[dx=1,Dx=0.5,dy=1,Dy=20]{->}(0,0)(9.2,5.2)
             \psdots[linewidth=0.1mm](0,0)(2,1)(4,1.74)(6,3.24)(8,4)(9,4.75)
             \psline[linewidth=0.5mm](0,0)(2,1)(4,1.74)(6,3.24)(8,4)(9,4.75)
             \rput(8,-0.8){\small Durée du parcours (en h)}
             \rput(1.3,5.3){\small Distance parcourue (en km)}
             \rput(-0.2,4.7){95}
             \psset{linecolor=violet,linewidth=0.4mm,linestyle=dashed}
                \psline{->}(2,0)(2,1)(0,1)
                \rput(-0.7,1){\textcolor{violet}{(b)}}
                \psline{->}(3.5,0)(3.5,1.6)(0,1.6)
                \rput(-0.7,1.6){\textcolor{violet}{(d)}}
                \psline{->}(4,0)(4,1.74)(0,1.74)
                \psline{->}(6,0)(6,3.24)(0,3.24)
                \psline[linestyle=solid]{|-|}(-0.5,1.74)(-0.5,3.24)
                \psline[linestyle=solid]{|-|}(4,-0.7)(6,-0.7)
                \rput(-0.7,2.49){\textcolor{violet}{(e)}}
                \psline{->}(9,0)(9,4.75)(0,4.75)
                \rput(-0.7,4.75){\textcolor{violet}{(g)}}
          \end{pspicture}
       \end{center} 
       \begin{enumerate}
          \item La distance parcourue en fonction de la durée est représentée par une fonction dont le support est une ligne brisée qui n'est pas une droite. Par conséquent : \par
             \con{la durée du parcours n'est pas proportionnelle à la distance parcourue.}
          \item Le point d'abscisse 1 a pour ordonnée 20 donc, \con{Quentin a parcouru \Lg[km]{20} en \Temps{;;;1}.}
          \item Une distance de \Lg[km]{20} en une heure correspond a une distance de \Lg[m]{20000} en \Temps{;;;;;3600}.
             $$v=\dfrac{d}{t} =\dfrac{\Lg[m]{20000}}{\Temps{;;;;;3600}} \approx\Vitesse[ms]{5,5556}.$$
             \con{La vitesse moyenne de Quentin a été d'environ \Vitesse[ms]{5,56} durant la première heure.}
          \item \Temps{;;;1;45} est égal à \Temps{;;;1,75}. Pour une abscisse de 1,75, la fonction a pour ordonnée environ 32. \par
             \con{Quentin a parcouru environ \Lg[km]{32} en \Temps{;;;1;45}}  
          \item En \Temps{;;;2}, Quentin à parcouru environ \Lg[km]{35}. \par
             En \Temps{;;;3}, Quentin à parcouru environ \Lg[km]{65}. Il a donc parcouru environ \Lg[km]{30} en \Temps{;;;1}, \par
             ce qui fait \con{une vitesse moyenne d'environ \Vitesse{30} durant la troisième heure.}
          \item La vitesse de Quentin durant la première heure est de $\Vitesse{20}$ alors qu'elle a été de $\Vitesse{30}$ lors de la troisième heure. Calculons le pourcentage d'évolution :
             $$p_e =\dfrac{v_f-v_i}{v_i}\times100 =\dfrac{\Vitesse{30}-\Vitesse{20}}{\Vitesse{20}}\times100 =50\,\%$$
             \con{La vitesse moyenne de Quentin lors de la troisième heure est supérieure de plus de 40\,\% à sa vitesse moyenne lors de la première heure.}
          \item Finalement, Quentin a parcouru \Lg[km]{95} en \Temps{;;;4,5}. \par
             $$v=\dfrac{d}{t} =\dfrac{\Lg[km]{95}}{\Temps{;;;4,5}} \approx\Vitesse[ms]{21,1111}.$$
          \con{La vitesse moyenne de Quentin a été d'environ \Vitesse[kmh]{21,11} durant sa sortie.}
       \end{enumerate} 
 \end{enumerate}