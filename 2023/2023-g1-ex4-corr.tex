{\bf Partie A}
\begin{enumerate}
   \item Le terrain est un rectangle $ABEC$ d'aire \Aire[m]{100}. \par
      Sa longueur $CE$ vaut : $CE =CD+DF+FE =\Lg[m]{6.5}+\Lg[m]{2}+\Lg[m]{4} =\Lg[m]{12.5}$. \par
      Sachant que $\mathcal{A}(ABEC) =CE\times CA$, alors $CA =\dfrac{\mathcal{A}(ABEC)}{CE} =\dfrac{\Aire[m]{100}}{\Lg[m]{12.5}} =\Lg[m]{8}$. \par
      \con{La longueur du segment $[CA]$ est égale à \Lg[m]{8}.}
   \item À l'échelle 1\,:\,80, on obtient les les mesures suivantes :
      \begin{center}
         \hautab{1.5}
         \begin{tabular}{|c|C{2}|C{2}|C{2}|C{2}|}
            \hline
            Longueur en cm dans la réalité & \Lg{1250} & \Lg{800} & \Lg{750} & \Lg{100} \\
            \hline
            Longueur en cm sur le plan & \Lg{15.625} & \Lg{10} & \Lg{9.375} & \Lg{1.25}\\
            \hline
         \end{tabular} $\downarrow \div 80$

         \medskip
         \begin{pspicture}(0,-0.3)(16,10.3)
            \psset{linecolor=violet}
            \psframe(0,0)(15.625,10)
            \psline(8.125,0)(0,10)
            \psline(10.625,0)(15.625,10)
            \psarc(9.375,0){1.25}{0}{180} 
            \rput(9.375,0){$\times$}
            \rput(3,3){\fbox{Zone 1}}
            \rput(14,3){\fbox{Zone 2}}
            \rput(9,6){\fbox{Zone 3}}
            \rput(-0.3,10.3){$A$}
            \rput(-0.3,-0.3){$C$}
            \rput(15.925,-0.3){$E$}
            \rput(15.925,10.3){$B$}
            \rput(8.125,-0.3){$D$}
            \rput(10.625,-0.3){$F$}
            \rput(9.375,-0.3){Entrée}
         \end{pspicture}
      \end{center}
   \item
      \begin{enumerate}
         \item Avec des mesures de longueur en mètres, on a : \par 
            \Pythagore[Racine,Unite={}]{DCA}{6.5}{8}{}
            \con{La longueur du segment $[AD]$ est égale à $\sqrt{106,25}$ m.}
         \item Soit $L$ la longueur de la bordure de la zone 1. \par
            $L =AD+DC+CA =\sqrt{106,25}\text{ m}+\Lg[m]{6.5}+\Lg[m]{8} \approx\Lg[m]{24,81}$. \par
            \con{La bordure autour de la zone 1 est d'environ \Lg[m]{25}.}
         \item $\Lg[m]{25} =7\times\Lg[m]{4}-\Lg[m]{3}$. \con{Il faudra donc sept rouleaux de bordures pour entourer la zone 1.}
      \end{enumerate}
   \item Les zones 1 et 2 sont des triangles (rectangles). Leur aire se calcule grâce à la formule $\mathcal{A} =\dfrac{\text{base}\times\text{hauteur}}{2}$. \par
      \begin{enumerate}
         \item $\mathcal{A}_1 =\dfrac{CD\times CA}{2} =\dfrac{\Lg[m]{6.5}\times\Lg[m]{8}}{2} =\Aire[m]{26}$. \con{La zone 1 a une aire de $\Aire[m]{26}$.}
         \item $\mathcal{A}_2 =\dfrac{EF\times EB}{2} =\dfrac{\Lg[m]{4}\times\Lg[m]{8}}{2} =\Aire[m]{16}$. \con{La zone 2 a une aire de $\Aire[m]{16}$.}
         \item Pour déterminer l'aire $\mathcal{A}_3$ de la zone 3, il faut soustraire les aires des zones 1, 2 et de l'entrée à l'aire totale. \par
            Aire de l'entrée : $\mathcal{A}_E =\dfrac{\pi\times R^2}{2} =\dfrac{\pi\times(\Lg[m]{1})^2}{2} =\dfrac{\pi}{2}\Aire[m]{}$. \par
            $\mathcal{A}_3 =\mathcal{A}_{ABEC}-\mathcal{A}_1-\mathcal{A}_2-\mathcal{A}_E =\Aire[m]{100}-\Aire[m]{26}-\Aire[m]{16}-\dfrac{\pi}{2}\Aire[m]{} =\left(58-\dfrac{\pi}{2}\right)\Aire[m]{}$. \par
            \con{La zone 3 a une aire de $\left(58-\dfrac{\pi}{2}\right)\Aire[m]{} \approx\Aire[m]{56,4}$.}
      \end{enumerate}
   \item $6\times\left(58-\dfrac{\pi}{2}\right) \approx338,6$. Dans la zone 3, on peut donc planter 338 pieds de fraisiers. \par
      Sachant qu'un pied donne \Masse{650} de fraise, on a alors : $338\times\Masse{650} =\Masse{219700} =\Masse[kg]{219,7}$ \par
      \con{Les élèves peuvent espérer récolter environ \Masse[kg]{220} de fraise en une année.}
\end{enumerate}

{\bf Partie B}

\begin{enumerate}
   \item On suppose ici que la confiture de fraise se prépare uniquement à partir de sucre et de fraises. \par
      Le sucre représente 55\,\% de la masse totale avant cuisson, ce qui veut dire que les fraises représentent 45\,\%.
      \begin{description}
         \item[Méthode 1] : Le sucre et les fraises sont dans un ratio de 55\,:\,45, ou encore 11\,:\,9 ce qui veut dire que pour \Masse[kg]{9} de fraises, on a \Masse[kg]{11} de sucre. \par
            Donc, pour \Masse[kg]{25} de fraise, soit $\dfrac{25}{9}$ fois plus de fraise, on a $\dfrac{25}{9}\times\Masse[kg]{11} \approx\Masse[kg]{30,6}$ de sucre.
         \item[Méthode 2] : Si les \Masse[kg]{25} de fraises représentent 45\,\% de la masse totale, alors la masse totale est égale à \par
            $\Masse[kg]{25}\times100\div45 \approx\Masse[kg]{55,55}$. \par
            Et 55\,\% de cette masse totale est égal à $\dfrac{55}{100}\times\Masse[kg]{55,55} \approx\Masse[kg]{30,6}$.
      \end{description}
      \con{Le directeur devra acheter \Masse[kg]{31} de sucre pour cette recette.}
   \item Si \Masse[kg]{3} de fraises permettent de faire \Capa{48} de confiture, \Masse[kg]{25} de fraises permettent de faire $\dfrac{\Masse[kg]{25}\times\Capa{4,8}}{\Masse[kg]{3}} =\Capa{40}$. \par
      \con{Avec \Masse[kg]{25} de fraises, on peut faire \Capa{40} de confiture.}
   \item Un pot cylindrique d'un diamètre de \Lg{8,4}, soit de rayon \Lg{4,2} et de hauteur \Lg{11} a pour volume  : \par
      $\mathcal{V} =\dfrac{\text{aire de la base}\times\text{hauteur}}{2} =\dfrac{\pi\times(\Lg{4,2})^2\times\Lg{11}}{2} =194,04\,\pi\,\Vol{}$. \par
      Or, $194,04\,\pi\,\Vol{} =0,19404\,\pi\,\Vol[dm]{} =0,19404\,\pi\,\Capa{}$. \par
      Les pots étant remplis au 8/9 de leur capacité, la capacité de confiture qu'il est possible de mettre dans un pot est donc de $\dfrac89\times0,19404\,\pi\,\Capa{} \approx0,17248\,\pi\,\Capa{}$. \par
      On a \Capa{40} de confiture et $\dfrac{\Capa{40}}{0,17248\,\pi\,\Capa{}} \approx73,82$. \con{Pour \Capa{40} de confiture, il faudra 74 pots.}
\end{enumerate}