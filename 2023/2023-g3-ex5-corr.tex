\begin{enumerate}
    \item 257 est un nombre entier naturel, ensemble inclus dans l'ensemble des nombres décimaux ($\mathbb{N}\in\mathbb{D}$), c'est donc un nombre décimal (on peut aussi l'écrire sous la forme d'une fraction décimale : $\dfrac{257}{10^0} =\dfrac{2570}{10^1}\dots$). \par
       \con{L'affirmation 1 est vraie.}
    \item $\dfrac73-8 =\dfrac73-\dfrac{24}{3} =-\dfrac{17}{3}$. Ce nombre, écrit en écriture fractionnaire, possède un numérateur et un dénominateur entiers, c'est donc bien un nombre rationnel. \par
       \con{L'affirmation 2 est vraie.}
    \item Soit $n$ un nombre entier quelconque. La somme de trois nombres entiers consécutifs s'écrit : \vskip-7mm
       \begin{align*}
          n+(n+1)+(n+2) &=n+n+1+n+2 \\
          &=3n+3 =3(n+1) \quad \text{qui est bien un multiple de 3}
       \end{align*}
       \con{L'affirmation 3 est vraie.}
    \item On commence par développer les deux membres de l'équation avant de simplifier puis de résoudre l'équation.
       \begin{align*}
          (x+1)(x-2) &=(x-3)(x+4) \\
          x^2-2x+x-2 &=x^2+4x-3x-12 \\
          \cancel{x^2}-x-2 &=\cancel{x^2}+x-12 \\
          -x-2 &=x-12 \\
          -2+12 &=x+x \\
          10 &=2x \\
          5 &=x
       \end{align*}
       La solution de l'équation est $x =5$ qui est un nombre entier. \par
       \con{L'affirmation 4 est vraie.}
    \item Une augmentation de 15\,\% correspond à un coefficient multiplicateur de $1+\dfrac{15}{100} =1,15$. \par
       Une augmentation de 10\,\% correspond à un coefficient multiplicateur de $1+\dfrac{10}{100} =1,10$. \par
       Donc, une augmentation de 15\,\% puis de 10\,\% correspond à un coefficient multiplicateur de $1,15\times1,10 =1,265$, soit une augmentation de 26,5\,\%, supérieure à une augmentation d'un quart (25\,\%). \par
       \con{L'affirmation 5 est fausse.}
    \item Une quadrilatère ayant un angle droit n'est pas nécessairement un rectangle (il faudrait pour cela qu'il en ait trois). \par
       Contre-exemple : \begin{pspicture}(0,0)(2,1.4) \pspolygon(0,0)(2,0)(1,1.3)(0,1) \psframe(0,0)(0.3,0.3) \end{pspicture} \par
       \con{L'affirmation 6 est fausse.}
    \item Un triangle rectangle possède un angle droit (90°). Or, un triangle équilatéral possède trois angles égaux à 60°. \newline
       Un triangle ne peut donc pas être à la fois rectangle et équilatéral. \par
       \con{L'affirmation 7 est fausse.}
    \item L'antécédent de 4 est le nombre $x$ tel que $f(x) =4$.
       \ResolEquation[Entier,Simplification]{-3}{1}{0}{4}
       \con{L'affirmation 8 est fausse.}
    \item Toutes les mesures de longueur sont en cm. \par
       \Pythagore[Reciproque]{ABC}{39}{12}{35}
       \con{L'affirmation 9 est fausse.}
    \item Multiplier les longueur par 3 (c'est à dire faire un agrandissement de facteur 3), revient à multiplier son aire par $3^2 =9$. \par
       \con{L'affirmation 10 est fausse.} 
 \end{enumerate}