\begin{enumerate}
    \item Si l'élève choisit le dé marqué d'un 3, il choisit 3 jetons. S'il choisit le dé marqué d'un 2, il en choisit 4. S'il choisit de passer son tour, il prend aucun jeton. \par
       \con{L'élève peut prendre 0, 3 ou 4 jetons.}
    \item Pour obtenir 3 jetons, il faut que  le plus grand (au sens large) des deux jetons soit égal à 3, puisque c'est un nombre impair, impossible à obtenir avec un dé de valeur plus petite. \par
       \con{Les tirages possibles pour obtenir un 3 sont : << obtenir 1 et 3 >>, << obtenir 2 et 3 >> ou << obtenir 3 et 3 >>.}
    \item On peut construire un tableau récapitulant toutes les issues possibles en indiquant le nombre de jetons qui peut être pris pour chacun des tirages et en distinguant les deux dés. \par
       Les dés sont équilibrés, il y a donc équiprobabilité de chaque << case >> et 36 issues possibles. On notera $(d_1;d_2)$ une issue possible où $d_1$ est le résultat de l'un des dés et $d_2$ le résultat du deuxième.
       \begin{center}
          {\hautab{1.3}
          \begin{tabular}{|*{7}{C{1.2}|}}
             \hline
             & \bf 1 & \bf 2 & \bf 3 & \bf 4 & \bf 5 & \bf 6 \\
             \hline
             \bf 1 & 1 & 2 & 2 ou 3 & 2 ou 4 & 2 ou 5 & 2 ou 6 \\
             \hline
             \bf 2 & 2 & 2 & 3 ou 4 & 4 & 4 ou 5 & 4 ou 6 \\
             \hline
             \bf 3 & 2 ou 3 & 3 ou 4 & 3 & 4 ou 6 & 5 ou 6 & 6 \\
             \hline
             \bf 4 & 2 ou 4 & 4 & 4 ou 6 & 4 & 5 ou 8 & 6 ou 8 \\
             \hline
             \bf 5 & 2 ou 5 & 4 ou 5 & 5 ou 6 & 5 ou 8 & 5 & 6 ou 10 \\
             \hline
             \bf 6 & 2 ou 6 & 4 ou 6 & 6 & 6 ou 8 & 6 ou 10 & 6 \\
             \hline
          \end{tabular}}
       \end{center}
       \begin{enumerate}
          \item Il y a deux issues possibles à l'événement << les nombres obtenus sont un 3 et un 2 >> : \par
             (2;3) ou (3;2) \par \smallskip
             $\mathcal{P}_a =\dfrac{\text{nombre de cas favorables}}{\text{nombre de cas possibles}} =\dfrac{2}{36} =\con{\dfrac{1}{18}}$.
          \item Il y a onze issues possibles à l'événement << au moins un des nombres obtenus est 3 >> : \par
             (3;1) ; (3;2) ; (3;3) ; (3;4) ; (3;5) ; (3;6) ; (1;3) ; (2;3) ; (4;3) ; (5;3) et (6;3). \par \smallskip 
             $\mathcal{P}_b =\dfrac{\text{nombre de cas favorables}}{\text{nombre de cas possibles}} =\con{\dfrac{11}{36}}$.
          \item Il y a treize issues possibles à l'événement << es nombres obtenus permettent de prendre 4 jetons >> : \par
             (1;4) ; (2;3) ; (2;4) ; (2;5) ; (2;6) ; (3;2) ; (3;4) ; (4;1) ; (4;2) ; (4;3) ; (4;4) ; (5;2) et (6;2). \par \smallskip 
             $\mathcal{P}_c =\dfrac{\text{nombre de cas favorables}}{\text{nombre de cas possibles}} =\con{\dfrac{13}{36}}$.
       \end{enumerate}
    \item Chaque lancer est indépendant du précédent, ce qui signifie que le résultat d'un lancer n'a aucune influence sur le suivant (on reprend les mêmes dés et on recommence identiquement). \par
       \con{La probabilité qu'il puisse prendre trois jetons au lancer suivant un lancer où il a pu prendre trois jetons est la même.}
    \item Avec un 1 et un 4, il a trois choix : \par
       -- Il passe sont tour, et garde ses 12 jetons. Il devra donc encore récupérer 3 jetons lors des tours suivants. \newline
       -- Il choisit de doubler la valeur du plus petit dé et de prendre 2 jetons, ce qui lui en fait 14. Il devra donc récupérer 1 seul jeton lors des tours suivants. O, il n'y a qu'un cas possible pour cela qui est l'éventualité (1;1). \newline 
       -- il choisit la valeur du plus grand dé et prend 4 jetons, ce qui lui en fait 16. Dans ce cas, il dépasse les 15 jetons demandés et a perdu. \par
       Étant donné que la probabilité de prendre 3 jetons au lancer d'après est plus élevée que celle d'en prendre un seul, \newline
       \con{il est préférable pour l'élève de passer son tour.}
 \end{enumerate}