\begin{enumerate}
    \setlength{\itemsep}{-1mm}
    \item L’entier 4 216 est-il un multiple de 17 ? Justifier.
    \item Guillaume veut revoir sa leçon en prenant son petit déjeuner. Malheureusement, il a renversé son chocolat sur sa feuille. Le chiffre des unités et la justification de l’exemple du maître, sont illisibles\dots
       \begin{center}
          \hskip-3cm \Large{2 29 \psellipse[fillstyle=solid,fillcolor=brown,linecolor=brown](0.25,0.2)(0.4,0.6) \qquad est divisible par 3 car \pscurve*[linecolor=brown](0,-0.5)(0.3,0.9)(1,1)(2,0.6)(4,0.7)(3,-0.5)(2,-0.3)(1,-0.2)(0.5,-0.4)}
       \end{center}
       \begin{enumerate}
          \setlength{\itemsep}{-1mm}
          \item Rappeler le critère de divisibilité par 3.
          \item Donner toutes les valeurs possibles du chiffre des unités, caché par la tâche située à gauche.
       \end{enumerate}
    \item On admet qu’un nombre entier n est divisible par 7 si et seulement si la différence entre son nombre de dizaines et le double de son chiffre des unités est un multiple 7, positif ou négatif. \newline 
       Par exemple, 294 est divisible par 7 car $29-4\times2=21$, et 21 est divisible par 7.
       \begin{enumerate}
          \setlength{\itemsep}{-1mm}
          \item En détaillant les étapes, vérifier que 413 est bien divisible par 7 en utilisant le critère indiqué ci-dessus.
          \item Le nombre 5 292 est-il divisible par 7 ? Répondre en appliquant, plusieurs fois si nécessaire, le critère précédent.
          \item Pour déterminer si 1 138 984 est divisible par 7, on utilise le critère précédent à l’aide d’un tableur. \newline
             On rappelle que la fonction \texttt{ENT} renvoie la partie entière d’un nombre. \vspace*{-5mm}
             \begin{center}
                \begin{pspicture}(0,0)(11,0)
                   \psframe[fillstyle=solid,fillcolor=white,linecolor=white](0.08,-3.82)(10.92,-0.18)
                \end{pspicture}
                \begin{Tableur}[LargeurUn=20mm,Largeur=20mm,Cellule=B1,Colonne=2,Ligne=1,Formule=ENT(A1/10)]
                   {\raggedleft} 1138984 & {\raggedleft 113898} & {\raggedleft 4} & \multicolumn{1}{r}{113890} \\
                   {\raggedleft} 113890 & {\raggedleft 11389} & {\raggedleft 0} & \multicolumn{1}{r}{11389} \\
                   {\raggedleft} 11389 & {\raggedleft 1138} & {\raggedleft 9} & \multicolumn{1}{r}{1120} \\
                   {\raggedleft} 1120 & {\raggedleft 112} & {\raggedleft 0} & \multicolumn{1}{r}{112} \\
                   {\raggedleft} 112 & {\raggedleft 11} & {\raggedleft 2} & \multicolumn{1}{r}{7} \\
                \end{Tableur}
             \end{center}
             Dans la cellule \texttt{B1} on a saisi la formule : << \texttt{= ENT(A1/10)} >>. \newline
             Observer la feuille de calcul puis indiquer des formules ayant pu être saisies dans les cellules \texttt{C1} et \texttt{D1} qui, étirées vers le bas de la feuille de calcul, permettent d’obtenir directement la feuille de calcul ci-dessus.
          \item Le nombre 1 138 984 est-il divisible par 7 ? Justifier en interprétant les résultats fournis par la feuille de calcul.
       \end{enumerate}
 \end{enumerate}