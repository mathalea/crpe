Nadia se prépare pour le cross organisé par son école dont le parcours, $ABIFCDE$, est représenté ci-dessous. \vspace*{-2mm}
\begin{enumerate}
   \setlength{\itemsep}{-1mm}
   \item Les droites $(AE)$ et $(BD)$ se coupent en $C$. Les droites $(AB)$, $(FI)$ et $(DE)$ sont parallèles. \newline
      $F\in[AC]$ et $I\in[BC]$. $ABC$ est un triangle rectangle en $A$. \newline
      $AB =\Lg[m]{300}$ ; $AC =\Lg[m]{400}$ ; $CD =\Lg[m]{1250}$ et $IC =\Lg[m]{350}$. \vspace*{-2mm}
      \begin{enumerate}
         \setlength{\itemsep}{-1mm}
         \item Déterminer la longueur $BC$.
         \item Déterminer les longueurs $IF$ et $CF$.
         \item Déterminer la longueur $ED$.
         \item Calculer la longueur du parcours $ABIFCDE$.
      \end{enumerate}
      \begin{center}
         \psset{unit=0.55}
         \small
         \begin{pspicture}(-5,-2.8)(13,8.2)
            \rput{-50}(0,5){
               \pstGeonode[PointSymbol=none,PosAngle={90,-45,60,135,-135,90,-60}]{C}(10,0){E}(10,7.5){D}(-4,0){A}(-4,-3){B}(-2.8,0){F}(-2.8,-2.1){I}
               \psframe[fillstyle=solid,fillcolor=lightgray](-4,0)(-3.7,-0.3)
               \psline[linewidth=0.5mm](-4,0)(-4,-3)(-2.8,-2.1)(-2.8,0)(0,0)(10,7.5)(10,0)
               \psline[linestyle=dashed](-2.8,-2.1)(0,0)(10,0)
               \psline[linestyle=dashed](-4,0)(-2.8,0)}
         \end{pspicture}
      \end{center}
   \item Quentin, un adolescent de 16 ans, fait du vélo. On a représenté ci-dessous la distance parcourue en fonction de la durée de parcours lors de sa dernière sortie.
      \begin{center}
         \psset{xunit=1.55,yunit=1.2}
         \begin{pspicture}(-0.5,-0.8)(9,5.4)
            \psgrid[subgriddiv=5, gridlabels=0pt,gridcolor=gray,subgridcolor=gray!80](0,0)(9.2,5.2)
            \psaxes[dx=1,Dx=0.5,dy=1,Dy=20]{->}(0,0)(9.2,5.2)
            \psdots[linewidth=0.1mm](0,0)(2,1)(4,1.74)(6,3.24)(8,4)(9,4.75)
            \psline[linewidth=0.5mm](0,0)(2,1)(4,1.74)(6,3.24)(8,4)(9,4.75)
            \psline[linestyle=dashed](0,4.75)(9,4.75)
            \rput(8,-0.8){\small Durée du parcours (en h)}
            \rput(1.3,5.3){\small Distance parcourue (en km)}
            \rput(-0.2,4.7){95}
         \end{pspicture}
      \end{center}
      \begin{enumerate}
         \setlength{\itemsep}{-1mm}
         \item La durée du parcours en heure est-elle proportionnelle à la distance parcourue en kilomètre ? Justifier. \par
         Les réponses aux questions suivantes seront données avec la précision permise par le graphique.
         \item Quelle distance a parcouru Quentin en 1h ?
         \item Déterminer la vitesse moyenne de Quentin durant la première heure, en mètre par seconde, avec un arrondi au centième.
         \item Quelle distance a parcouru Quentin en 1h45 ?
         \item Estimer la vitesse de Quentin durant la troisième heure de son parcours, en kilomètre par heure.
         \item Peut-on affirmer que sa vitesse moyenne lors de la troisième heure est supérieure de plus de 40\,\% à sa vitesse moyenne lors de la première heure ? Justifier.
         \item Quelle est la vitesse moyenne de Quentin lors de cette sortie, en kilomètre par heure, avec un arrondi au centième.
      \end{enumerate}
\end{enumerate}