Pour chaque affirmation, indiquer si elle est vraie ou fausse. Justifier.
   \begin{description}
      \setlength{\itemsep}{-1mm}
      \item[Affirmation 1] : << 257 est un nombre décimal. >> \smallskip
      \item[Affirmation 2] : << $\dfrac73-8$ est un nombre rationnel. >> \smallskip
      \item[Affirmation 3] : << la somme de trois nombres entiers consécutifs est toujours un multiple de 3. >>
      \item[Affirmation 4] : << l’équation $(x+1)(x-2) =(x-3)(x+4)$ admet un nombre entier comme solution. >>
      \item[Affirmation 5] : << augmenter une quantité de 15\,\% puis de 10\,\% revient à l’augmenter de son quart. >>
      \item[Affirmation 6] : << un quadrilatère ayant un angle droit est un rectangle. >>
      \item[Affirmation 7] : << un triangle rectangle peut-être équilatéral. >>
      \item[Affirmation 8] : << par la fonction $f$ définie par $f(x) =-3x+1$, l'antécédent de 4 est $-11$. >>
      \item[Affirmation 9] : << le triangle $ABC$ schématisé ci-dessous est rectangle. >>
         \begin{center}
            \begin{pspicture}(0,-0.2)(4,1.7)
               \pstTriangle[PointSymbol=none]{A}(4,0){C}(3.5,1.4){B}
               \footnotesize
                  \pstLabelAB[offset=-3mm]{A}{C}{\Lg{39}}
                  \pstLabelAB{B}{C}{\Lg{12}}
                  \pstLabelAB{A}{B}{\Lg{35}}
            \end{pspicture}
         \end{center} 
      \item[Affirmation 10] : << si on multiplie par 3 les longueurs des côtés d’un rectangle, alors son aire est également multipliée par 3. »
   \end{description}