Un professeur des écoles, organise avec sa classe de CM1 une randonnée à vélo. Le parcours BCEFCDB est représenté ci-contre.
   \begin{center}
      \begin{pspicture}(-4,-2.3)(3.5,1.5)
         \pstGeonode[CurveType=polygon,PosAngle={135,90,0,40,220},PointSymbol=none](-3.75,0){B}(0,0){C}(3.4,0){E}(3;28.07){F}(4.25;208.07){D}
         \pstLabelAB[offset=-3mm]{D}{C}{8,5 km}
         \pstLabelAB[offset=-3mm]{C}{E}{6,8 km}
         \pstLabelAB[offset=3mm]{B}{C}{7,5 km}
         \pstLabelAB[offset=3mm]{F}{E}{3,2 km}
         \pstLabelAB[offset=3mm]{C}{F}{6 km}
         \pstRightAngle{D}{B}{C}
      \end{pspicture}
   \end{center}
   \begin{enumerate}
      \setlength{\itemsep}{-1mm}
      \item Montrer que l’angle $\widehat{CFE}$ est droit.
      \item Déterminer la longueur totale du parcours.
      \item Sachant que la vitesse moyenne du groupe est de 14~km/h, la classe fera-t-elle le parcours en moins de 2~h~45~min ? Justifier la réponse.
   \end{enumerate}