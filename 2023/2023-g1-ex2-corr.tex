\begin{enumerate}
    \item
       \begin{enumerate}
          \item Soit $S$ la somme totale partagée, on a : $a =\dfrac14S$ et $b =\dfrac13S$. \par \smallskip
             Il reste donc $S-\dfrac14S-\dfrac13S =\dfrac{12S-3S-4S}{12} =\dfrac{5}{12}S$. \par \smallskip
             C et D se partagent ce reste équitablement donc : $c =d =\dfrac12\times\dfrac{5}{12}S =\dfrac{5}{24}S$. \par
             \con{$c$ représente $\dfrac{5}{24}$ de la somme totale.} \smallskip
          \item $d =\Prix{55}$ donc, $c =d =\Prix{55}$. \par \smallskip
             Or, $c =\Prix{55} =\dfrac{5}{24}S$, soit $S =\dfrac{24}{5}\times\Prix{55} =\Prix{264}$. \par \smallskip
             $b =\dfrac13S =\dfrac13\times\Prix{264} =\Prix{88}$. \par \smallskip
             $a =\dfrac14S =\dfrac14\times\Prix{264} =\Prix{66}$. \par \medskip
             \con{$a, b$ et $c$ valent respectivement \Prix{66}, \Prix{88} et \Prix{55}.}
       \end{enumerate}
    \item L'énoncé nous fournit les données suivantes :
       \begin{itemize}
          \item[(E1)] $e+f+g+h =s$
          \item[(E2)] $e =3f$
          \item[(E3)] $g+h =\dfrac13s$
          \item[(E4)] $g =h$
       \end{itemize}
       De (E3) et (E4), on en déduit que $h+h =\dfrac13s$, soit $2h=\dfrac13s$, ou encore $h=\dfrac16s$. \par
       (E4) nous dit alors que $g =h =\dfrac16s$. \par
       De (E1) et (E2), on en déduit que : $e+f+g+h =3f+f+\dfrac13s =s$. Soit $4f =s-\dfrac13s =\dfrac23s$ et $f =\dfrac14\times\dfrac2{3} =\dfrac16s$. \par
       (E2) implique que $e =3f =3\times\dfrac16s =\dfrac36s =\dfrac12s$. \par \smallskip
       \con{E reçoit la moitié de la somme $s$, et F, G et H reçoivent chacun un sixième de $s$.}
 \end{enumerate}