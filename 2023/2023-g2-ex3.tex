Voici deux programmes de calcul : \par \smallskip
   \hskip18mm
   \begin{minipage}[t]{7.5cm}
      \hskip12mm \texttt{Programme A} \par \medskip
      \begin{Scratch}[Echelle=0.8]
         Place Drapeau;
         Place Demander("Donner un nombre");
         Place MettreVar("a",OvalCap("réponse"));
         Place MettreVar("b",OpAdd(OpMul("2",OvalVar("a")),"5"));
         Place MettreVar("c",OpSous(OpMul("5",OvalVar("a")),"4"));
         Place MettreVar("d",OpMul(OvalVar("b"),OvalVar("c")));
         Place Dire(OvalVar("d"));
      \end{Scratch}
   \end{minipage}
   \qquad
   \begin{minipage}[t]{6cm}
      \hskip8mm \texttt{Programme B} \par \medskip
      \colorbox{white}{\parbox{5cm}{Choisir un nombre \par
      Prendre son double \par
      Ajouter 5 \par
      Calculer le carré du résultat \par
      Retourner le résultat trouvé}}
   \end{minipage}
   \begin{enumerate}
      \setlength{\itemsep}{-1mm}
      \item Montrer que si l’utilisateur saisit le nombre 2, alors le programme A retourne le nombre 54.
      \item Calculer le résultat obtenu avec le programme A si le nombre saisi par l’utilisateur est 1,15.
      \item Pour quel(s) nombre(s) de départ le programme A retourne-t-il le nombre 0 ?
      \item 
         \begin{enumerate}
            \setlength{\itemsep}{-1mm}
            \item Si l’utilisateur saisit le nombre 3, quel résultat le programme B retourne-t-il ?
            \item Si l’utilisateur saisit le nombre 4 , quel résultat le programme B retourne-t-il ?
         \end{enumerate}
      \item On détermine les résultats suivants retournés par le programme B à l’aide d’une feuille de calcul automatisé.
         \begin{center}
            \begin{Tableur}[Bandeau=false,Colonnes=10,Largeur=9mm]
               \RowStyle[rowcolor=white]{} 0 & 1 & 2 & 3 & 4 & 5 & 6 & 7 & 8 & \\
               \RowStyle[rowcolor=white]{} 25 & 49 & 81 & 121 & 169 & 225 & 289 & 361 & 441 & \\
               \RowStyle[rowcolor=white]{} & & & & & & & & & \\
            \end{Tableur}
         \end{center}
         \begin{enumerate}
            \setlength{\itemsep}{-1mm}
            \item Quelle cellule du tableur permet de retrouver la réponse à la question 4.(a) ci-dessus ?
            \item Quelle formule a pu être saisie dans la cellule \texttt{A2} de la feuille de calcul automatisé afin de la copier-glisser sur la ligne 2 ?
         \end{enumerate}
      \item 
         \begin{enumerate}
            \setlength{\itemsep}{-1mm}
            \item Pour quel nombre de départ le programme B retourne-t-il le nombre zéro ?
            \item Ce nombre de départ est-il rationnel ? Justifier.
            \item Ce nombre de départ est-il décimal ? Justifier.
         \end{enumerate}
      \item Pour quel(s) nombre(s) de départ le programme A retourne-t-il le même résultat que le programme B ?
      \end{enumerate}