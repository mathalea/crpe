\begin{enumerate}
    \item Les mesures de longueurs sont en km.
       \Pythagore[Reciproque]{CFE}{6.8}{6}{3.2} 
       \con{L'angle $\widehat{CFE}$ est droit.}
    \item Pour calculer la longueur totale du parcours, en km, il nous faut connaître la longueur du segment $[BD]$. \par
       \Pythagore[Soustraction,Exact,Unite={}]{DBC}{8.5}{7.5}{}
       On a alors : $DB+BC+CE+EF+FC+CD =\Lg[km]{4}+\Lg[km]{7,5}+\Lg[km]{6,8}+\Lg[km]{3,2}+\Lg[km]{6}+\Lg[km]{8,5} =\Lg[km]{36}$. \par
       \con{Le parcours total mesure \Lg[km]{36}.}
    \item Si on convertit la durée donnée en durée décimale, on obtient $\Temps{;;;2;45} =\Temps{;;;2,75}$.
       \begin{description}
          \item[Méthode 1] : 
             À une vitesse de \Vitesse{14} et un temps de \Temps{;;;2,75}, la distance parcourue serait de \par
             $\Temps{;;;2,75}\times\Vitesse{14} =\Lg[km]{38,5}$. \par
             Comme le parcours ne mesure que 36 km, la classe mettra moins de 2h 45 min pour l'effectuer.
          \item[Méthode 2] : On a la formule $v =\dfrac{d}{t}$, soit $t =\dfrac{d}{v}$. Donc, $t =\dfrac{\Lg[km]{36}}{\Vitesse{14}} \approx\Temps{;;;2,57}$. \par
             Or, $\Temps{;;;2,57}<\Temps{;;;2,75}$.
       \end{description}
       \con{La classe effectuera le parcours en moins de \Temps{;;;2;45}.}
 \end{enumerate}