\begin{enumerate}
    \item $4\,216 =17\times248$ donc, \con{l'entier 4\,216 est un multiple de 17.}
    \item
       \begin{enumerate}
          \item \con{Un nombre est divisible par 3 si, et seulement si, la somme de ses chiffres est divisible par 3 (ou multiple de 3).}
          \item $2+2+9 =13$. Le chiffre des unités est un nombre entier compris entre 0 et 9, la somme des chiffres est donc un nombre entier entre 13 et 22. Les seuls entiers possibles divisibles par 3 sont 15, 18 ou 21. \par
             \con{Le chiffre des unités peut être 2, 5 ou 8.}
       \end{enumerate}
    \item 
       \begin{enumerate}
          \item Le nombre de dizaine de 413 est 41 car $413 =41\times10+3$ et le chiffre des unités de 413 est 3. \par
             Or, $41-3\times2 =35$ et $35 =7\times5$ est divisible par 7. D'après le critère, \con{413 est divisible par 7.}
          \item Le nombre de dizaines de 5 292 est 529 et son chiffre des unités est 2. De plus, $529-2\times2 =525$. \par
             Le nombre de dizaines de 525 est 52 et son chiffre des unités est 5. De plus, $52-5\times2 =42$. \par
             Or, $42 =7\times6$ est divisible par 7. D'après le critère, \con{5 292 est divisible par 7.}
          \item Dans la cellule \texttt{C1}, on a pu saisir la formule \con{\texttt{=A1-10*B1}}. \par
             Dans la cellule \texttt{D1}, on a pu saisir la formule \con{\texttt{=B1-C1*2}}. \par
          \item Les nombres de colone D donnent successivement le résultat du critère cité ci-dessus. \par
             Le dernier nombre en \texttt{D5} donne 7. Or, $7 =7\times1$ est divisible par 7 donc, \con{le nombre 1 138 984 est divisible par 7.}
       \end{enumerate}
 \end{enumerate}