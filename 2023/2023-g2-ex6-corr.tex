\begin{enumerate}
    \item Application de la règle pour le produit de 6 par 8 : \par
       -- Avec le pied et la main gauches, on lève les 5 orteils et 1 doigt, représentant ainsi le 6. \par
       -- Avec le pied et la main droites, on lève les 5 orteils et 3 doigts, représentant ainsi le 8. \par
       La somme du nombre de doigts levés nous indique un nombre de dizaines, on a donc $(1+3)$ dizaines, soit 4 dizaines. \par
       Le produit des doigts baissés nous indique un nombre d’unités, on a donc $(4\times2)$ unités, soit 8 unites. \par
       \con{On obtient alors le nombre 48.}
    \item 
       \begin{enumerate}
          \item $(5-g)$ est égal à 5 (nombre de doigts d'une main) auquel on soustrait le nombre de doigts de la main gauche levés, \con{$(5-g)$ représente le nombre de doigts de la main gauche baissés.} \par
             Par un raisonnement analogue, \con{$(5-d)$ représente le nombre de doigts de la main droite baissés.} \par
          \item On développe le premier terme de l'égalité :
             \begin{align*}
                (5+g)(5+d) &=5\times5+5\times d+g\times5 +g\times d \\
                &=25+5d+5g+gd
             \end{align*}
             On développe le deuxième terme de l'égalité :
             \begin{align*}
                10\,(g+d)+(5-g)(5-d) &=10\times g+10\times d+5\times5+5\times(-d)+(-g)\times5+(-g)\times(-d) \\
                &=10g+10d+25-5d-5g+gd \\
                &=25+5d+5g+gd
             \end{align*}
             Les deux membres de l'égalités sont égaux, on a donc bien \par
             \con{$(5+g)(5+d) =10\,(g+d)+(5-g)(5-d)$.}
          \item $(5+g)$ et $(5+g)$ correspondent respectivement aux deux facteurs du produit demandé, << 5 >> étant le nombre d'orteils relevés. $(5+g)(5+d)$ représente donc le produit recherché. \par
             $10\,(g+d)$ est le nombre de doigts relevés, multiplié par 10, il s'agit donc du nombre de dizaines. \par
             D'après la question 2.(a), $(5-g)(5-d)$ correspond au produit du nombre de doigts baissés dans chaque main, il s'agit du nombre d'unités. \par
             Conclusion : \con{L'égalité est vérifiée et est la traduction algébrique de la règle énoncée, cette règle est donc valide.}
       \end{enumerate}
 \end{enumerate}
 