Un professeur des écoles d’une classe de CE1 présente à ses élèves une règle de calcul qui permet de déterminer avec ses dix doigts et ses dix orteils le produit de deux nombres entiers compris entre 5 et 10 en utilisant les résultats des tables appris précédemment. Il s’appuie sur l’exemple suivant :
\begin{center}
   \fbox{\parbox{13cm}{\it Effectuons $6\times7$. \par
      -- Avec le pied et la main gauches, on lève les 5 orteils et 1 doigt, représentant ainsi le 6. \par
      -- Avec le pied et la main droites, on lève les 5 orteils et 2 doigts, représentant ainsi le 7. \par
      Pour le calcul on ne regarde que les mains et on procède de la manière suivante : \par
      la somme du nombre de doigts levés nous indique un nombre de dizaines, le produit des doigts baissés nous indique un nombre d’unités. \par
      Ici on a : $(1+2)$ dizaines et $(4\times3)$ unités, soit encore 3 dizaines et 12 unités. \par
      On obtient donc le nombre 42.}}
\end{center}
\begin{enumerate}
   \setlength{\itemsep}{-1mm}
   \item Appliquer cette règle pour calculer le produit $6\times8$.
   \item On note $g$ le nombre de doigts levés de la main gauche et $d$ le nombre de doigts levés de la main droite.
      \begin{enumerate}
         \item Que représentent dans ce contexte les nombres $(5-g)$ et $(5-d)$ ?
         \item Démontrer l’égalité : $(5+g)(5+d) =10\,(g+d)+(5-g)(5-d)$.
         \item Conclure quand à la validité de la règle de calcul.
      \end{enumerate}
\end{enumerate}