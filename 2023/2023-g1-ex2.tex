\begin{enumerate}
    \setlength{\itemsep}{-1mm}
    \item Quatre personnes A, B, C, D se partagent une somme d’argent. On appelle $a$, $b$, $c$ et $d$ les montants respectivement reçus par A, B, C et D. On sait par ailleurs que :
       \begin{itemize}
          \item $a$ représente $\dfrac14$ de la somme totale ;
          \item $b$ représente $\dfrac13$ de la somme totale ;
          \item C et D se partagent ce qui reste en prenant chacun le même montant.
       \end{itemize}
       \begin{enumerate}
          \item Déterminer la proportion que représente c par rapport à la somme totale.
          \item D reçoit 55 \euro. Déterminer les valeurs de $a$, $b$ et $c$.
       \end{enumerate}
    \item Quatre personnes E, F, G, H se partagent une somme d’argent $s$. On appelle $e$, $f$, $g$ et $h$ les montants respectivement reçus par E, F, G et H. On sait par ailleurs que :
       \begin{itemize}
          \item E perçoit le triple de F ;
          \item $g+h$ représente $\dfrac13$ de la somme totale ;
          \item $g =h$.
       \end{itemize}
       Exprimer la part de chacun en fonction de s.
 \end{enumerate}