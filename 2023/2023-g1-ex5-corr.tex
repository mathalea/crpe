\begin{enumerate}
    \item La somme des mesures des trois murs qui porteront la frise est égale à : $2\times\Lg[m]{8,8}+\Lg[m]{7} =\Lg[m]{24,6} =\Lg{2460}$. \par
       Une feuille au format A4 a une longueur de $\Lg{29,7}$. Or, $\dfrac{\Lg{2460}}{\Lg{29,7}} \approx82,82$. \par
       \con{Il faudra 83 feuilles pour réaliser la frise.}
    \item Entre la chute de l'Empire romain et l'année 2023, il y a 1547 années. \par
       Ces 1547 années doivent êtres représentées sur une frise de \Lg{2460}. Or, $\dfrac{\Lg{2460}}{1547} \approx\Lg{1,59}$. \par
       \con{Une année sur la frise chronologique est représentée par environ \Lg{1,6}.}
    \item
       \begin{enumerate}
          \item \con{Dans la cellule \texttt{C2}, on pourrait proposer \texttt{=(B2-476)*1,6}}, ou \texttt{=(B2-476)*1,59} pour plus de précision.
          \item La formule \texttt{ENT(C2/29,7)+1} donne \con{le numéro de la feuille sur laquelle se trouvera la date} en prenant comme origine de la numérotation la feuille n°1 qui se situe au coin du mur NO.  
       \end{enumerate}
    \item $1492-476 =1016$. L'événement << accostage de Christophe Colomb sur le continent américain >>, daté de 1492 est à 1016 années de l'origine de la frise. \par
       Sachant qu'une année est représentée par \Lg{1,59}, 1016 années seront représentées par environ \par
       $1016\times\Lg{1,59} =\Lg{1615,44} =\Lg[m]{16,15}$. \par
       Or, le coin au SE est à $\Lg[m]{8,8}+\Lg[m]{7} =\Lg[m]{15,8}$ du départ de la frise. \par
       \con{Cet événement est sur le mur sud.}
 \end{enumerate}